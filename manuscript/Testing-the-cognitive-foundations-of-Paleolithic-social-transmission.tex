% !TeX program = pdfLaTeX
\documentclass[smallextended]{svjour3}       % onecolumn (second format)
%\documentclass[twocolumn]{svjour3}          % twocolumn
%
\smartqed  % flush right qed marks, e.g. at end of proof
%
\usepackage{amsmath}
\usepackage{graphicx}
\usepackage[utf8]{inputenc}

\usepackage[hyphens]{url} % not crucial - just used below for the URL
\usepackage{hyperref}
\providecommand{\tightlist}{%
  \setlength{\itemsep}{0pt}\setlength{\parskip}{0pt}}

%
% \usepackage{mathptmx}      % use Times fonts if available on your TeX system
%
% insert here the call for the packages your document requires
%\usepackage{latexsym}
% etc.
%
% please place your own definitions here and don't use \def but
% \newcommand{}{}
%
% Insert the name of "your journal" with
% \journalname{myjournal}
%

%% load any required packages here



% Pandoc citation processing
\newlength{\csllabelwidth}
\setlength{\csllabelwidth}{3em}
\newlength{\cslhangindent}
\setlength{\cslhangindent}{1.5em}
% for Pandoc 2.8 to 2.10.1
\newenvironment{cslreferences}%
  {}%
  {\par}
% For Pandoc 2.11+
\newenvironment{CSLReferences}[3] % #1 hanging-ident, #2 entry sp
 {% don't indent paragraphs
  \setlength{\parindent}{0pt}
  % turn on hanging indent if param 1 is 1
  \ifodd #1 \everypar{\setlength{\hangindent}{\cslhangindent}}\ignorespaces\fi
  % set line spacing
  % set entry spacing
  \ifnum #2 > 0
  \setlength{\parskip}{#3\baselineskip}
  \fi
 }%
 {}
\usepackage{calc} % for \widthof, \maxof
\newcommand{\CSLBlock}[1]{#1\hfill\break}
\newcommand{\CSLLeftMargin}[1]{\parbox[t]{\maxof{\widthof{#1}}{\csllabelwidth}}{#1}}
\newcommand{\CSLRightInline}[1]{\parbox[t]{\linewidth}{#1}}
\newcommand{\CSLIndent}[1]{\hspace{\cslhangindent}#1}


\begin{document}

\title{Testing the motor and cognitive foundations of Paleolithic social
transmission \thanks{Grants or other notes about the article that should
go on the front page should be placed here. General acknowledgments
should be placed at the end of the article.} }



\author{  Justin Pargeter 1, 2 \and  Megan Beney Kilgore 3 \and  Cheng
Liu 3 \and  Dietrich Stout 3 \and  }


\institute{
        Justin Pargeter 1, 2 \at
     1. Department of Anthropology, New York University, New York, NY,
USA; 2. Palaeo-Research Institute, University of Johannesburg, Auckland
Park, South Africa \\
     \email{\href{mailto:justin.pargeter@nyu.edu}{\nolinkurl{justin.pargeter@nyu.edu}}}  %  \\
%             \emph{Present address:} of F. Author  %  if needed
    \and
        Megan Beney Kilgore 3 \at
     3. Department of Anthropology, Emory University, Atlanta, GA,
USA \\
     \email{\href{mailto:megan.elizabeth.beney@emory.edu}{\nolinkurl{megan.elizabeth.beney@emory.edu}}}  %  \\
%             \emph{Present address:} of F. Author  %  if needed
    \and
        Cheng Liu 3 \at
     3. Department of Anthropology, Emory University, Atlanta, GA,
USA \\
     \email{\href{mailto:raylc1996@outlook.com}{\nolinkurl{raylc1996@outlook.com}}}  %  \\
%             \emph{Present address:} of F. Author  %  if needed
    \and
        Dietrich Stout 3 \at
     3. Department of Anthropology, Emory University, Atlanta, GA,
USA \\
     \email{\href{mailto:dwstout@emory.edu}{\nolinkurl{dwstout@emory.edu}}}  %  \\
%             \emph{Present address:} of F. Author  %  if needed
    \and
    }

\date{Received: date / Accepted: date}
% The correct dates will be entered by the editor


\maketitle

\begin{abstract}
Stone tools provide key evidence of human cognitive evolution but remain
difficult to interpret. Toolmaking skill-learning in particular has been
understudied even though: 1) the most salient cognitive demands of
toolmaking should occur during learning, and 2) variation in learning
aptitude would have provided the raw material for any past selection
acting on tool making ability. However, we actually know very little
about the cognitive prerequisites of learning under different
information transmission conditions that may have prevailed during the
Paleolithic. This paper presents results from a pilot experimental study
to trial new experimental methods for studying the effect of learning
conditions and individual differences on Oldowan flake-tool making skill
acquisition. We trained 23 participants for 2 hours to make stone flakes
under two different instructional conditions (observation only
vs.~direct active teaching) employing appropriate raw materials,
practice time, and real human interaction. Participant performance was
evaluated through analysis of the stone artifacts produced. Performance
was compared both across experimental groups and with respect to
individual participant differences in grip strength, motor accuracy, and
cognitive function measured for the study. Our results show aptitude to
be associated with fluid intelligence in a verbally instructed group and
with a tendency to use social information in an observation-only group.
These results have implications for debates surrounding the cumulative
nature of human culture, the relative contributions of knowledge and
know-how for stone tool making, and the role of evolved psychological
mechanisms in ``high fidelity'' transmission of information,
particularly through imitation and teaching.
\\
\keywords{
        Oldowan \and
        Stone toolmaking \and
        Social learning \and
        Individual variation \and
        Cognitive aptitudes \and
        Motor skills \and
    }


\end{abstract}


\def\spacingset#1{\renewcommand{\baselinestretch}%
{#1}\small\normalsize} \spacingset{1}


\hypertarget{intro}{%
\section{Introduction}\label{intro}}

Stone tools have long been seen as a key source of evidence for
understanding human behavioral and cognitive evolution (Darwin 1871;
Oakley 1949; Washburn 1960). Pathbreaking attempts to infer specific
cognitive capacities from this evidence largely focused on the basic
requirements of tool production (Isaac 1976; Wynn 1979; Gowlett 1984;
Wynn and Coolidge 2004). More recently, increasing attention has been
directed to the processes and demands of stone tool making skill
acquisition (Roux, Bril, and Dietrich 1995; Stout 2002; Stout et al.
2005; Geribàs, Mosquera, and Vergès 2010; Nonaka, Bril, and Rein 2010;
Stout et al. 2011; Putt, Woods, and Franciscus 2014; Hecht, Gutman,
Khreisheh, et al. 2015; Duke and Pargeter 2015; Morgan et al. 2015;
Stout and Khreisheh 2015; Lombao, Guardiola, and Mosquera 2017; Putt et
al. 2017; Cataldo, Migliano, and Vinicius 2018; Putt, Wijeakumar, and
Spencer 2019; Pargeter and Shea 2019; Pargeter et al. 2020). This is
motivated by the expectation that the most salient cognitive demands of
tool making should occur during learning rather than routine expert
performance (Stout and Khreisheh 2015) and by interest in the relevance
of different social learning mechanisms such as imitation (Rein, Nonaka,
and Bril 2014; Stout et al. 2019), emulation (Tehrani and Riede 2008;
Wilkins 2018), and language (Ohnuma, Aoki, and Akazawa 1997; Putt,
Woods, and Franciscus 2014; Morgan et al. 2015; Lombao, Guardiola, and
Mosquera 2017; Putt et al. 2017; Cataldo, Migliano, and Vinicius 2018)
to the reproduction of Paleolithic technologies.

Studies investigating these questions have used a range of different
experimental designs (e.g., varying technological goals/instructions,
training times, raw materials, live vs.~recorded instruction,
lithic/skill assessment metrics, pseudo-knapping tasks etc.) and reached
disparate conclusions regarding the neurocognitive and social
foundations of skill acquisition. It is plausible that these discordant
results reflect actual diversity in how humans acquire and master stone
tool making skills. However, this failure of results to generalize
across artificial experimental manipulations
(\href{cf.\%20Yarkoni\%202020}{cf}. Yarkoni 2020) also raises doubts
regarding the external validity (Eren et al. 2016) of conclusions with
respect to real-world Paleolithic learning contexts. To address this, we
conducted an exploratory study that draws on lessons from previous
research in an attempt to balance the pragmatic and theoretical
tradeoffs inherent in experimental studies of stone knapping skill
acquisition (Pargeter and Shea 2019; Stout and Khreisheh 2015).

Learning real-world skills like stone knapping is highly demanding of
time and materials and difficult to control experimentally without
sacrificing generalizability to real world conditions. Prior efforts
have attempted to navigate these challenges by using various
combinations of 1) inauthentic raw materials that are less expensive,
easier to standardize, and/or easier to knap, 2) video-recorded
instruction that is uniform across participants and less demanding of
experimenter time, 3) short learning periods, 4) small sample sizes, and
5) single learning conditions. The difficulty of interpreting results
from this growing literature led Stout and Khreisheh (2015: 870,
emphasis original) to call for ``studies with sufficient sample sizes to
manipulate learning conditions (e.g.~instruction, motivation) and assess
individual variation (e.g.~performance, psychometrics, neuroanatomy)
that \emph{also} have realistic learning periods.'' The current study
attempts to strike a viable balance between these demands by
investigating early-stage learning of a relatively simple technology
(least effort, ``Oldowan,'' flake production (Reti 2016; Shea 2016)
under two instructional conditions while collecting data on individual
differences in strength, coordination, cognition, social learning,
self-control, and task engagement. Unlike any previous study, this
allows us to address the likelihood that group effects of training
conditions might be impacted by interactions with individual participant
differences in aptitude, motivation, or learning style.

We focus on early stage learning because it has been found to be
relatively rapid, variable across individuals, and predictive of later
outcomes (Pargeter and Shea 2019; Stout and Khreisheh 2015; Putt,
Wijeakumar, and Spencer 2019), and thus provides a reasonable
expectation of generating meaningful data on skill and learning
variation while minimizing training costs. Moreover, understanding the
minimum training times necessary to detect changes in tool making skill
will help archaeologists design more realistic and cost-effective
experiments. To further manage costs, we limited our study to only two
learning conditions (observation only vs.~active teaching). This targets
a key controversy in human evolution, namely the origins of teaching and
language (Gärdenfors and Högberg 2017; Morgan et al. 2015), while
avoiding highly artificial manipulations of dubious relevance to
real-world Paleolithic learning. These choices allowed us to invest more
in other aspects of research design that we identified as theoretically
important, including measurement of individual differences in cognition
and behavior, inclusion of an in-person, fully interactive teaching
condition, and use of naturalistic raw materials. Sample size remained
small in this internally funded exploratory study but could easily be
scaled up at funding levels typical of pre- and post-doctoral research
grants in archaeology.

\hypertarget{individual-differences}{%
\subsection{\texorpdfstring{\textbf{Individual
Differences}}{Individual Differences}}\label{individual-differences}}

``\emph{The many slight differences\ldots{} being observed in the
individuals of the same species inhabiting the same confined locality,
may be called individual differences\ldots{} These individual
differences are of the highest importance to us, for they are often
inherited \ldots{} and they thus afford materials for natural selection
to act on and accumulate\ldots{}}'' (Darwin 1859, Chapter 2)

Individuals vary in aptitude and learning style for particular skills
(Jonassen and Grabowski 1993) but this has largely been ignored in
studies of knapping skill acquisition, which have instead focused on
group effects of different experimental conditions. There are good
pragmatic reasons for this, as individual difference studies typically
require larger sample sizes and additional data collection. However,
overlooking these distinctions is not ideal since individual differences
can provide valuable insight into the mechanisms, development, and
evolution of cognition and behavior (Boogert et al. 2018). In
particular, patterns of association between cognitive traits and
behavioral performance can be used to test hypotheses about the
cognitive demands of learning particular skills and the likely targets
of natural selection acting on aptitude. More prosaically, individual
differences can introduce an unexamined and uncontrolled source of
variation in group level results. This is especially true in the
relatively small ``samples of convenience'' typical of experimental
archaeology.

While testing hypotheses in evolutionary cognitive archaeology remains a
considerable challenge (Wynn 2017), investigation of individual
variation in modern research participants represents one promising
direction. For any particular behavior of archaeological interest, it is
expected that standing variation in modern populations should remain
relevant to normal variation in learning aptitude. The presence of trait
variation without impact on learning aptitude would provide strong
evidence against the plausibility of the proposed evolutionary
relationship. An absence of variation (i.e., past fixation and rigorous
developmental canalization) is not expected given the known variability
of human brains and cognition (Sherwood and Gómez-Robles 2017; Barrett
2020). Any confirmatory findings of trait-aptitude correspondence would
then have the testable implication that humans should be evolutionarily
derived along the same dimension (e.g. Hecht, Gutman, Bradley, et al.
2015).

To date, a small number of ``neuroarchaeological'' studies have reported
associations between individual knapping performance and brain structure
or physiological responses. Hecht et al. (2015) reported
training-related changes in white matter integrity (fractional
anisotropy {[}FA{]}) that correlated with individual differences in
practice time and striking accuracy change. The regional patterning of
FA changes also varied across individuals, with only those individuals
who displayed early increases in FA under the right ventral precentral
gyrus (premotor cortex involved in movement planning and guidance)
showing striking accuracy improvement over the training period. Putt et
al. (2019) similarly found that the proportion of flakes to shatter
produced by individuals during handaxe making correlated with dorsal
precentral gyrus (motor cortex) activation. Pargeter et al. (2020) used
a flake prediction paradigm (modeled after Nonaka, Bril, and Rein 2010)
to confirm that striking force and accuracy are important determinants
of handaxe-making success. These findings all point to the central role
of perceptual-motor systems (Stout and Chaminade 2007) and coordination
(Roux, Bril, and Dietrich 1995) in knapping skill acquisition. In
addition, Putt et al. (2019) also found successful flake production to
be associated with prefrontal (working memory/cognitive control)
activation and Stout et al.~(2015) found that prefrontal activation
correlated with success at a strategic judgement (platform selection)
task which in turn was predictive of success at out-of-scanner handaxe
production. Such investigations are thus starting to chart out the more
specific contributions of different neural systems to particular aspects
of knapping skill acquisition. To date, however, the
cognitive/functional interpretation of systems identified in this manner
has largely relied on informal reverse inference (reasoning backward
from observed activations to inferred mental processes) from published
studies of other tasks that activated the same regions, an approach
which is widely regarded as problematic (Poldrack 2011).

Here we take a more direct, psychometric approach to measuring
individual differences in perceptual-motor coordination and cognition.
Psychometric instruments (e.g., tasks, questionnaires) are designed to
assess variation in cognitive traits and states, such as fluid
intelligence, working memory, attention, motivation, and personality,
that have been of theoretical interest to cognitive archaeologists
(e.g., Wynn and Coolidge 2016). It is thus surprising that they have
been almost entirely neglected in experimental studies of knapping
skill. In the only published example we are aware of, Pargeter et al.
(2019) reported significant effects of variation in planning and problem
solving (Tower of London test (Shallice, Broadbent, and Weiskrantz
1982)) and cognitive set shifting (Wisconsin Card Sort test (Grant and
Berg 1948)) on early stage handaxe learning. Of course, cognition is not
the only thing that can affect knapping performance. Flake prediction
experiments highlight the importance of regulating movement
speed/accuracy trade-offs (Nonaka, Bril, and Rein 2010; Pargeter et al.
2020) and studies of muscle recruitment (Marzke et al. 1998) and manual
pressure (Williams-Hatala et al. 2018; Alastair J. M. Key and Dunmore
2018) during knapping highlight basic strength requirements. Along these
lines, Key and Lycett (2019) found that individual differences in hand
size, shape, and especially grip strength were better predictors of
force loading during stone tool use than were attributes of the tools
themselves. However, we are unaware of any such studies of biometric
influences on variation in knapping success. Finally, the time and
effort demands of knapping skill acquisition suggest that differences in
personality (e.g., self-control and ``grit'' (Pargeter and Shea 2019),
motivation (Stout 2002), and social vs.~individual learning strategies
(Miu et al. 2020) might also affect learning outcomes. We are again
unaware of any previous studies that have assessed such effects. In this
study, we assessed all participants with a battery of tests including
grip strength, movement speed/accuracy, spatial working memory, fluid
intelligence, self-control, tendency to use social information, and
motivation/engagement with the tool making task. We were particularly
interested in the possibility that these variables might not only impact
learning generally, but might also have different effects under
different learning conditions.~

\hypertarget{teaching-language-and-tool-making}{%
\subsection{\texorpdfstring{\textbf{Teaching, Language, and Tool
Making}}{Teaching, Language, and Tool Making}}\label{teaching-language-and-tool-making}}

\emph{A creature that learns to make tools to a complex pre-existing
pattern\ldots must have the kind of abstracting mind that would be of
high selective value in facilitating the development of the ability to
communicate such skills by the necessary verbal acts.} (Montagu 1976:
267)

Possible links between tool making and language have been a subject of
speculation for nearly 150 years (Engles 2003, {[}1873{]}), if not
longer (Hewes 1993), although compelling empirical tests have remained
elusive. Over 25 years ago, Toth and Schick (1993) suggested that
experiments teaching modern participants to make stone tools in verbal
and non-verbal conditions could test the importance of language in the
social reproduction of Paleolithic technologies. Ohnuma et al. (1997)
were the first to implement this suggestion in a study of Levallois
flake production, followed by more recent studies of handaxe making
(Putt, Woods, and Franciscus 2014; Putt et al. 2017) and simple flake
production (Morgan et al. 2015; Cataldo, Migliano, and Vinicius 2018;
Lombao, Guardiola, and Mosquera 2017). This reflects recent interest in
the hypothesis that language might be an adaptation for teaching (e.g.,
Laland 2017; Stout and Chaminade 2012). Teaching and learning demands of
Paleolithic tool making would thus provide evidence of selective
contexts favoring language evolution (Stout 2010; Morgan et al. 2015;
Montagu 1976).

\hfill\break

\hypertarget{raw-materials-and-knapping-skill}{%
\subsection{\texorpdfstring{\textbf{Raw materials and knapping
skill}}{Raw materials and knapping skill}}\label{raw-materials-and-knapping-skill}}

\hypertarget{materials-and-methods}{%
\section{\texorpdfstring{\textbf{Materials and
Methods}}{Materials and Methods}}\label{materials-and-methods}}

\hypertarget{participants}{%
\subsection{\texorpdfstring{\textbf{Participants}}{Participants}}\label{participants}}

\hypertarget{study-visit}{%
\subsection{\texorpdfstring{\textbf{Study
Visit}}{Study Visit}}\label{study-visit}}

\hypertarget{individual-difference-measures}{%
\subsection{\texorpdfstring{\textbf{Individual Difference
Measures}}{Individual Difference Measures}}\label{individual-difference-measures}}

\hypertarget{stone-tool-making}{%
\subsection{\texorpdfstring{\textbf{Stone Tool
Making}}{Stone Tool Making}}\label{stone-tool-making}}

\hypertarget{raw-materials}{%
\subsubsection{\texorpdfstring{\textbf{Raw
Materials}}{Raw Materials}}\label{raw-materials}}

\hypertarget{experimental-conditions}{%
\subsubsection{\texorpdfstring{\textbf{Experimental
Conditions}}{Experimental Conditions}}\label{experimental-conditions}}

\hypertarget{lithic-analysis}{%
\subsection{\texorpdfstring{\textbf{Lithic
Analysis}}{Lithic Analysis}}\label{lithic-analysis}}

\hypertarget{statistical-analyses}{%
\subsection{\texorpdfstring{\textbf{Statistical
Analyses}}{Statistical Analyses}}\label{statistical-analyses}}

\hypertarget{results}{%
\section{\texorpdfstring{\textbf{Results}}{Results}}\label{results}}

\hypertarget{principal-component-analyses}{%
\subsection{\texorpdfstring{\textbf{Principal Component
analyses}}{Principal Component analyses}}\label{principal-component-analyses}}

\hypertarget{flake-size-and-shape}{%
\subsubsection{\texorpdfstring{\textbf{Flake size and
shape}}{Flake size and shape}}\label{flake-size-and-shape}}

\hypertarget{lithic-flaking-performance-measures}{%
\subsubsection{\texorpdfstring{\textbf{Lithic flaking performance
measures}}{Lithic flaking performance measures}}\label{lithic-flaking-performance-measures}}

\hypertarget{do-trained-untrained-and-expert-knappers-perform-differently}{%
\subsection{\texorpdfstring{\textbf{Do trained, untrained, and expert
knappers perform
differently?}}{Do trained, untrained, and expert knappers perform differently?}}\label{do-trained-untrained-and-expert-knappers-perform-differently}}

\hypertarget{does-trainingpractice-time-impact-flaking-performance}{%
\subsection{\texorpdfstring{\textbf{Does training/practice time impact
flaking
performance?}}{Does training/practice time impact flaking performance?}}\label{does-trainingpractice-time-impact-flaking-performance}}

\hypertarget{do-individual-differences-in-motor-skill-and-psychometric-measures-predict-flaking-performance}{%
\subsection{\texorpdfstring{\textbf{Do individual differences in motor
skill and psychometric measures predict flaking
performance?}}{Do individual differences in motor skill and psychometric measures predict flaking performance?}}\label{do-individual-differences-in-motor-skill-and-psychometric-measures-predict-flaking-performance}}

\hypertarget{model-1-individual-differences-and-quantity-flaking}{%
\subsubsection{\texorpdfstring{\textbf{Model 1: Individual differences
and quantity
flaking}}{Model 1: Individual differences and quantity flaking}}\label{model-1-individual-differences-and-quantity-flaking}}

\hypertarget{model-2-individual-differences-and-quality-flaking}{%
\subsubsection{\texorpdfstring{\textbf{Model 2: Individual differences
and quality
flaking}}{Model 2: Individual differences and quality flaking}}\label{model-2-individual-differences-and-quality-flaking}}

\hypertarget{discussion}{%
\section{\texorpdfstring{\textbf{Discussion}}{Discussion}}\label{discussion}}

\hypertarget{conclusions}{%
\section{\texorpdfstring{\textbf{Conclusions}}{Conclusions}}\label{conclusions}}

\hypertarget{acknowledgments}{%
\section{\texorpdfstring{\textbf{Acknowledgments}}{Acknowledgments}}\label{acknowledgments}}

\hypertarget{references}{%
\section*{\texorpdfstring{\textbf{References}}{References}}\label{references}}
\addcontentsline{toc}{section}{\textbf{References}}

\hypertarget{refs}{}
\begin{CSLReferences}{1}{0}
\leavevmode\hypertarget{ref-barrett2020}{}%
Barrett, H. Clark. 2020. {``Towards a Cognitive Science of the Human:
Cross-Cultural Approaches and Their Urgency.''} \emph{Trends in
Cognitive Sciences} 24 (8): 620--38.
\url{https://doi.org/10.1016/j.tics.2020.05.007}.

\leavevmode\hypertarget{ref-boogert2018}{}%
Boogert, Neeltje J., Joah R. Madden, Julie Morand-Ferron, and Alex
Thornton. 2018. {``Measuring and Understanding Individual Differences in
Cognition.''} \emph{Philosophical Transactions of the Royal Society B:
Biological Sciences} 373 (1756): 20170280.
\url{https://doi.org/10.1098/rstb.2017.0280}.

\leavevmode\hypertarget{ref-cataldo2018}{}%
Cataldo, Dana Michelle, Andrea Bamberg Migliano, and Lucio Vinicius.
2018. {``Speech, Stone Tool-Making and the Evolution of Language.''}
\emph{PLOS ONE} 13 (1): e0191071.
\url{https://doi.org/10.1371/journal.pone.0191071}.

\leavevmode\hypertarget{ref-darwin1859}{}%
Darwin, Charles. 1859. \emph{On the Origin of Species by Means of
Natural Selection, or, The Preservation of Favoured Races in the
Struggle for Life}. 1st ed. London: John Murray.

\leavevmode\hypertarget{ref-darwin1871}{}%
---------. 1871. \emph{The Descent of Man, and Selection in Relation to
Sex}. 1st ed. London: John Murray.

\leavevmode\hypertarget{ref-duke2015}{}%
Duke, Hilary, and Justin Pargeter. 2015. {``Weaving Simple Solutions to
Complex Problems: An Experimental Study of Skill in Bipolar
Cobble-Splitting.''} \emph{Lithic Technology} 40 (4): 349--65.
\url{https://doi.org/10.1179/2051618515Y.0000000016}.

\leavevmode\hypertarget{ref-engles2003}{}%
Engles, Friedrich. 2003. {``The Part Played by Labour in the Transition
from Ape to Man.''} In, edited by Robert C. Scharff and Val Dusek,
71--77. London: Blackwell.

\leavevmode\hypertarget{ref-eren2016}{}%
Eren, Metin I., Stephen J. Lycett, Robert J. Patten, Briggs Buchanan,
Justin Pargeter, and Michael J. O'Brien. 2016. {``Test, Model, and
Method Validation: The Role of Experimental Stone Artifact Replication
in Hypothesis-Driven Archaeology.''} \emph{Ethnoarchaeology: Journal of
Archaeological, Ethnographic and Experimental Studies} 8 (2): 103--36.
\url{https://doi.org/10.1080/19442890.2016.1213972}.

\leavevmode\hypertarget{ref-guxe4rdenfors2017}{}%
Gärdenfors, Peter, and Anders Högberg. 2017. {``The Archaeology of
Teaching and the Evolution of Homo Docens.''} \emph{Current
Anthropology} 58 (2): 188--208. \url{https://doi.org/10.1086/691178}.

\leavevmode\hypertarget{ref-geribuxe0s2010}{}%
Geribàs, Núria, Marina Mosquera, and Josep Maria Vergès. 2010. {``What
Novice Knappers Have to Learn to Become Expert Stone Toolmakers.''}
\emph{Journal of Archaeological Science} 37 (11): 2857--70.
\url{https://doi.org/10.1016/j.jas.2010.06.026}.

\leavevmode\hypertarget{ref-gowlett1984}{}%
Gowlett, John A. J. 1984. {``Mental Abilities of Early Man: A Look at
Some Hard Evidence.''} \emph{Higher Education Quarterly} 38 (3):
199--220. \url{https://doi.org/10.1111/j.1468-2273.1984.tb01387.x}.

\leavevmode\hypertarget{ref-grant1948}{}%
Grant, David A., and Esta Berg. 1948. {``A Behavioral Analysis of Degree
of Reinforcement and Ease of Shifting to New Responses in a Weigl-Type
Card-Sorting Problem.''} \emph{Journal of Experimental Psychology} 38
(4): 404--11. \url{https://doi.org/10.1037/h0059831}.

\leavevmode\hypertarget{ref-hecht2015b}{}%
Hecht, Erin E., David A. Gutman, Bruce A. Bradley, Todd M. Preuss, and
Dietrich Stout. 2015. {``Virtual dissection and comparative connectivity
of the superior longitudinal fasciculus in chimpanzees and humans.''}
\emph{NeuroImage} 108 (March): 124--37.
\url{https://doi.org/10.1016/j.neuroimage.2014.12.039}.

\leavevmode\hypertarget{ref-hecht2015a}{}%
Hecht, Erin E., David. A. Gutman, Nada Khreisheh, S. V. Taylor, J.
Kilner, A. A. Faisal, Bruce A. Bradley, T. Chaminade, and D. Stout.
2015. {``Acquisition of Paleolithic toolmaking abilities involves
structural remodeling to inferior frontoparietal regions.''} \emph{Brain
Structure \& Function} 220 (4): 2315--31.
\url{https://doi.org/10.1007/s00429-014-0789-6}.

\leavevmode\hypertarget{ref-hewes1993}{}%
Hewes, Gordon W. 1993. {``A History of Speculation on the Relation
Between Tools and Language.''} In, edited by Kathleen R. Gibson and Tim
Ingold, 20--31. Cambridge: Cambridge University Press.

\leavevmode\hypertarget{ref-isaac1976}{}%
Isaac, Glynn L. 1976. {``Stages of Cultural Elaboration in the
Pleistocene: Possible Archaeological Indicators of the Development of
Language Capabilities.''} \emph{Annals of the New York Academy of
Sciences} 280 (1): 275--88.
\url{https://doi.org/10.1111/j.1749-6632.1976.tb25494.x}.

\leavevmode\hypertarget{ref-jonassen1993}{}%
Jonassen, David H., and Barbara L. Grabowski. 1993. \emph{Handbook of
Individual Differences, Learning, and Instruction}. Hillsdale, NJ:
Lawrence Erlbaum,.

\leavevmode\hypertarget{ref-key2019}{}%
Key, A. J. M., and S. J. Lycett. 2019. {``Biometric Variables Predict
Stone Tool Functional Performance More Effectively Than Tool-Form
Attributes: A Case Study in Handaxe Loading Capabilities.''}
\emph{Archaeometry} 61 (3): 539--55.
\url{https://doi.org/10.1111/arcm.12439}.

\leavevmode\hypertarget{ref-key2018}{}%
Key, Alastair J. M., and Christopher J. Dunmore. 2018. {``Manual
Restrictions on Palaeolithic Technological Behaviours.''} \emph{PeerJ} 6
(August): e5399. \url{https://doi.org/10.7717/peerj.5399}.

\leavevmode\hypertarget{ref-laland2017}{}%
Laland, Kevin N. 2017. {``The Origins of Language in Teaching.''}
\emph{Psychonomic Bulletin \& Review} 24 (1): 225--31.
\url{https://doi.org/10.3758/s13423-016-1077-7}.

\leavevmode\hypertarget{ref-lombao2017}{}%
Lombao, D., M. Guardiola, and M. Mosquera. 2017. {``Teaching to Make
Stone Tools: New Experimental Evidence Supporting a Technological
Hypothesis for the Origins of Language.''} \emph{Scientific Reports} 7
(1): 1--14. \url{https://doi.org/10.1038/s41598-017-14322-y}.

\leavevmode\hypertarget{ref-marzke1998}{}%
Marzke, Mary W., N. Toth, K. Schick, S. Reece, B. Steinberg, K. Hunt, R.
L. Linscheid, and K.-N. An. 1998. {``EMG Study of Hand Muscle
Recruitment During Hard Hammer Percussion Manufacture of Oldowan
Tools.''} \emph{American Journal of Physical Anthropology} 105 (3):
315--32.
\url{https://doi.org/10.1002/(SICI)1096-8644(199803)105:3\%3C315::AID-AJPA3\%3E3.0.CO;2-Q}.

\leavevmode\hypertarget{ref-miu2020}{}%
Miu, Elena, Ned Gulley, Kevin N. Laland, and Luke Rendell. 2020.
{``Flexible Learning, Rather Than Inveterate Innovation or Copying,
Drives Cumulative Knowledge Gain.''} \emph{Science Advances} 6 (23):
eaaz0286. \url{https://doi.org/10.1126/sciadv.aaz0286}.

\leavevmode\hypertarget{ref-montagu1976}{}%
Montagu, Ashley. 1976. {``Toolmaking, Hunting, and the Origin of
Language.''} \emph{Annals of the New York Academy of Sciences} 280 (1):
266--74. \url{https://doi.org/10.1111/j.1749-6632.1976.tb25493.x}.

\leavevmode\hypertarget{ref-morgan2015}{}%
Morgan, T. J. H., N. T. Uomini, L. E. Rendell, L. Chouinard-Thuly, S. E.
Street, H. M. Lewis, C. P. Cross, et al. 2015. {``Experimental Evidence
for the Co-Evolution of Hominin Tool-Making Teaching and Language.''}
\emph{Nature Communications} 6 (1): 6029.
\url{https://doi.org/10.1038/ncomms7029}.

\leavevmode\hypertarget{ref-nonaka2010}{}%
Nonaka, Tetsushi, Blandine Bril, and Robert Rein. 2010. {``How Do Stone
Knappers Predict and Control the Outcome of Flaking? Implications for
Understanding Early Stone Tool Technology.''} \emph{Journal of Human
Evolution} 59 (2): 155--67.
\url{https://doi.org/10.1016/j.jhevol.2010.04.006}.

\leavevmode\hypertarget{ref-oakley1949}{}%
Oakley, Kenneth P. 1949. \emph{Man the Toolmaker}. London: Trustees of
the British Museum.

\leavevmode\hypertarget{ref-ohnuma1997}{}%
Ohnuma, Katsuhiko, Kenichi Aoki, and And Takeru Akazawa. 1997.
{``Transmission of Tool-Making Through Verbal and Non-Verbal
Commu-Nication: Preliminary Experiments in Levallois Flake
Production.''} \emph{Anthropological Science} 105 (3): 159--68.
\url{https://doi.org/10.1537/ase.105.159}.

\leavevmode\hypertarget{ref-pargeter2020}{}%
Pargeter, Justin, Nada Khreisheh, John J. Shea, and Dietrich Stout.
2020. {``Knowledge Vs. Know-How? Dissecting the Foundations of Stone
Knapping Skill.''} \emph{Journal of Human Evolution} 145 (August):
102807. \url{https://doi.org/10.1016/j.jhevol.2020.102807}.

\leavevmode\hypertarget{ref-pargeter2019}{}%
Pargeter, Justin, and John J. Shea. 2019. {``Going Big Versus Going
Small: Lithic Miniaturization in Hominin Lithic Technology.''}
\emph{Evolutionary Anthropology: Issues, News, and Reviews} 28 (2):
72--85. \url{https://doi.org/10.1002/evan.21775}.

\leavevmode\hypertarget{ref-poldrack2011}{}%
Poldrack, Russell A. 2011. {``Inferring Mental States from Neuroimaging
Data: From Reverse Inference to Large-Scale Decoding.''} \emph{Neuron}
72 (5): 692--97. \url{https://doi.org/10.1016/j.neuron.2011.11.001}.

\leavevmode\hypertarget{ref-putt2017}{}%
Putt, Shelby S., Sobanawartiny Wijeakumar, Robert G. Franciscus, and
John P. Spencer. 2017. {``The Functional Brain Networks That Underlie
Early Stone Age Tool Manufacture.''} \emph{Nature Human Behaviour} 1
(6): 1--8. \url{https://doi.org/10.1038/s41562-017-0102}.

\leavevmode\hypertarget{ref-putt2019}{}%
Putt, Shelby S., Sobanawartiny Wijeakumar, and John P. Spencer. 2019.
{``Prefrontal Cortex Activation Supports the Emergence of Early Stone
Age Toolmaking Skill.''} \emph{NeuroImage} 199 (October): 57--69.
\url{https://doi.org/10.1016/j.neuroimage.2019.05.056}.

\leavevmode\hypertarget{ref-putt2014}{}%
Putt, Shelby S., Alexander D. Woods, and Robert G. Franciscus. 2014.
{``The Role of Verbal Interaction During Experimental Bifacial Stone
Tool Manufacture.''} \emph{Lithic Technology} 39 (2): 96--112.
\url{https://doi.org/10.1179/0197726114Z.00000000036}.

\leavevmode\hypertarget{ref-rein2014}{}%
Rein, Robert, Tetsushi Nonaka, and Blandine Bril. 2014. {``Movement
Pattern Variability in Stone Knapping: Implications for the Development
of Percussive Traditions.''} \emph{PLOS ONE} 9 (11): e113567.
\url{https://doi.org/10.1371/journal.pone.0113567}.

\leavevmode\hypertarget{ref-reti2016}{}%
Reti, Jay S. 2016. {``Quantifying Oldowan Stone Tool Production at
Olduvai Gorge, Tanzania.''} \emph{PLOS ONE} 11 (1): e0147352.
\url{https://doi.org/10.1371/journal.pone.0147352}.

\leavevmode\hypertarget{ref-roux1995}{}%
Roux, Valentine, Blandine Bril, and Gilles Dietrich. 1995. {``Skills and
Learning Difficulties Involved in Stone Knapping: The Case of
Stone{-}Bead Knapping in Khambhat, India.''} \emph{World Archaeology} 27
(1): 63--87. \url{https://doi.org/10.1080/00438243.1995.9980293}.

\leavevmode\hypertarget{ref-shallice1982}{}%
Shallice, Timothy, Donald Eric Broadbent, and Lawrence Weiskrantz. 1982.
{``Specific Impairments of Planning.''} \emph{Philosophical Transactions
of the Royal Society of London. B, Biological Sciences} 298 (1089):
199--209. \url{https://doi.org/10.1098/rstb.1982.0082}.

\leavevmode\hypertarget{ref-shea2016}{}%
Shea, John J. 2016. \emph{Stone Tools in Human Evolution: Behavioral
Differences Among Technological Primates}. Cambridge: Cambridge
University Press. \url{https://doi.org/10.1017/9781316389355}.

\leavevmode\hypertarget{ref-sherwood2017}{}%
Sherwood, Chet C., and Aida Gómez-Robles. 2017. {``Brain Plasticity and
Human Evolution.''} \emph{Annual Review of Anthropology} 46 (1):
399--419. \url{https://doi.org/10.1146/annurev-anthro-102215-100009}.

\leavevmode\hypertarget{ref-stout2002}{}%
Stout, Dietrich. 2002. {``Skill and Cognition in Stone Tool Production:
An Ethnographic Case Study from Irian Jaya.''} \emph{Current
Anthropology} 43 (5): 693--722. \url{https://doi.org/10.1086/342638}.

\leavevmode\hypertarget{ref-stout2010}{}%
---------. 2010. {``Possible Relations Between Language and Technology
in Human Evolution.''} In, edited by April Nowell and Iain Davidson,
159184. Boulder, CO: University Press of Colorado.

\leavevmode\hypertarget{ref-stout2007}{}%
Stout, Dietrich, and Thierry Chaminade. 2007. {``The Evolutionary
Neuroscience of Tool Making.''} \emph{Neuropsychologia} 45 (5):
1091--1100.
\url{https://doi.org/10.1016/j.neuropsychologia.2006.09.014}.

\leavevmode\hypertarget{ref-stout2012}{}%
---------. 2012. {``Stone Tools, Language and the Brain in Human
Evolution.''} \emph{Philosophical Transactions of the Royal Society B:
Biological Sciences} 367 (1585): 75--87.
\url{https://doi.org/10.1098/rstb.2011.0099}.

\leavevmode\hypertarget{ref-stout2015}{}%
Stout, Dietrich, and Nada Khreisheh. 2015. {``Skill Learning and Human
Brain Evolution: An Experimental Approach.''} \emph{Cambridge
Archaeological Journal} 25 (4): 867--75.
\url{https://doi.org/10.1017/S0959774315000359}.

\leavevmode\hypertarget{ref-stout2011}{}%
Stout, Dietrich, Richard Passingham, Christopher Frith, Jan Apel, and
Thierry Chaminade. 2011. {``Technology, expertise and social cognition
in human evolution.''} \emph{The European Journal of Neuroscience} 33
(7): 1328--38. \url{https://doi.org/10.1111/j.1460-9568.2011.07619.x}.

\leavevmode\hypertarget{ref-stout2005}{}%
Stout, Dietrich, Jay Quade, Sileshi Semaw, Michael J. Rogers, and Naomi
E. Levin. 2005. {``Raw Material Selectivity of the Earliest Stone
Toolmakers at Gona, Afar, Ethiopia.''} \emph{Journal of Human Evolution}
48 (4): 365--80. \url{https://doi.org/10.1016/j.jhevol.2004.10.006}.

\leavevmode\hypertarget{ref-stout2019}{}%
Stout, Dietrich, Michael J. Rogers, Adrian V. Jaeggi, and Sileshi Semaw.
2019. {``Archaeology and the Origins of Human Cumulative Culture: A Case
Study from the Earliest Oldowan at Gona, Ethiopia.''} \emph{Current
Anthropology} 60 (3): 309--40. \url{https://doi.org/10.1086/703173}.

\leavevmode\hypertarget{ref-tehrani2008}{}%
Tehrani, Jamshid J., and Felix Riede. 2008. {``Towards an Archaeology of
Pedagogy: Learning, Teaching and the Generation of Material Culture
Traditions.''} \emph{World Archaeology} 40 (3): 316--31.
\url{https://doi.org/10.1080/00438240802261267}.

\leavevmode\hypertarget{ref-toth1993}{}%
Toth, Nicholas, and Kathy Schick. 1993. {``Early Stone Industries and
Inferences Regarding Language and Cognition.''} In, edited by Kathleen
R. Gibson and Tim Ingold, 346362. Cambridge: Cambridge University Press.

\leavevmode\hypertarget{ref-washburn1960}{}%
Washburn, Sherwood L. 1960. {``Tools and Human Evolution.''}
\emph{Scientific American} 203 (3): 62--75.
\url{https://doi.org/10.1038/scientificamerican0960-62}.

\leavevmode\hypertarget{ref-wilkins2018}{}%
Wilkins, Jayne. 2018. {``The Point Is the Point: Emulative Social
Learning and Weapon Manufacture in the Middle Stone Age of South
Africa.''} In, edited by Michael J. O'Brien, Briggs Buchanan, and Metin
I. Eren, 153--74. Cambridge, MA: The MIT Press.

\leavevmode\hypertarget{ref-williams-hatala2018}{}%
Williams-Hatala, Erin Marie, Kevin G. Hatala, McKenzie Gordon, Alastair
Key, Margaret Kasper, and Tracy L. Kivell. 2018. {``The Manual Pressures
of Stone Tool Behaviors and Their Implications for the Evolution of the
Human Hand.''} \emph{Journal of Human Evolution} 119 (June): 14--26.
\url{https://doi.org/10.1016/j.jhevol.2018.02.008}.

\leavevmode\hypertarget{ref-wynn1979}{}%
Wynn, Thomas. 1979. {``The Intelligence of Later Acheulean Hominids.''}
\emph{Man} 14 (3): 371--91. \url{https://doi.org/10.2307/2801865}.

\leavevmode\hypertarget{ref-wynn2017}{}%
---------. 2017. {``Evolutionary Cognitive Archaeology.''} In, edited by
Thomas Wynn and Frederick Coolidge, 120. Oxford: Oxford University
Press.

\leavevmode\hypertarget{ref-wynn2004}{}%
Wynn, Thomas, and Frederick L. Coolidge. 2004. {``The expert Neandertal
mind.''} \emph{Journal of Human Evolution} 46 (4): 467--87.
\url{https://doi.org/10.1016/j.jhevol.2004.01.005}.

\leavevmode\hypertarget{ref-wynn2016}{}%
---------. 2016. {``Archeological Insights into Hominin Cognitive
Evolution.''} \emph{Evolutionary Anthropology: Issues, News, and
Reviews} 25 (4): 200--213. \url{https://doi.org/10.1002/evan.21496}.

\leavevmode\hypertarget{ref-yarkoni2020}{}%
Yarkoni, Tal. 2020. {``The Generalizability Crisis.''} \emph{Behavioral
and Brain Sciences}, 1--37.
\url{https://doi.org/10.1017/S0140525X20001685}.

\end{CSLReferences}

\bibliographystyle{spmpsci}
\bibliography{bibliography.bib}

\end{document}
