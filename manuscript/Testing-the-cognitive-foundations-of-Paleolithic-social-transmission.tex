% !TeX program = pdfLaTeX
\documentclass[smallextended]{svjour3}       % onecolumn (second format)
%\documentclass[twocolumn]{svjour3}          % twocolumn
%
\smartqed  % flush right qed marks, e.g. at end of proof
%
\usepackage{amsmath}
\usepackage{graphicx}
\usepackage[utf8]{inputenc}

\usepackage[hyphens]{url} % not crucial - just used below for the URL
\usepackage{hyperref}
\providecommand{\tightlist}{%
  \setlength{\itemsep}{0pt}\setlength{\parskip}{0pt}}

%
% \usepackage{mathptmx}      % use Times fonts if available on your TeX system
%
% insert here the call for the packages your document requires
%\usepackage{latexsym}
% etc.
%
% please place your own definitions here and don't use \def but
% \newcommand{}{}
%
% Insert the name of "your journal" with
% \journalname{myjournal}
%

%% load any required packages here



% Pandoc citation processing
\newlength{\csllabelwidth}
\setlength{\csllabelwidth}{3em}
\newlength{\cslhangindent}
\setlength{\cslhangindent}{1.5em}
% for Pandoc 2.8 to 2.10.1
\newenvironment{cslreferences}%
  {}%
  {\par}
% For Pandoc 2.11+
\newenvironment{CSLReferences}[3] % #1 hanging-ident, #2 entry sp
 {% don't indent paragraphs
  \setlength{\parindent}{0pt}
  % turn on hanging indent if param 1 is 1
  \ifodd #1 \everypar{\setlength{\hangindent}{\cslhangindent}}\ignorespaces\fi
  % set line spacing
  % set entry spacing
  \ifnum #2 > 0
  \setlength{\parskip}{#3\baselineskip}
  \fi
 }%
 {}
\usepackage{calc} % for \widthof, \maxof
\newcommand{\CSLBlock}[1]{#1\hfill\break}
\newcommand{\CSLLeftMargin}[1]{\parbox[t]{\maxof{\widthof{#1}}{\csllabelwidth}}{#1}}
\newcommand{\CSLRightInline}[1]{\parbox[t]{\linewidth}{#1}}
\newcommand{\CSLIndent}[1]{\hspace{\cslhangindent}#1}


\begin{document}

\title{Testing the motor and cognitive foundations of Paleolithic social
transmission \thanks{Grants or other notes about the article that should
go on the front page should be placed here. General acknowledgments
should be placed at the end of the article.} }



\author{  Justin Pargeter 1, 2 \and  Megan Beney Kilgore 3 \and  Cheng
Liu 3 \and  Dietrich Stout 3 \and  }


\institute{
        Justin Pargeter 1, 2 \at
     1. Department of Anthropology, New York University, New York, NY,
USA; 2. Palaeo-Research Institute, University of Johannesburg, Auckland
Park, South Africa \\
     \email{\href{mailto:justin.pargeter@nyu.edu}{\nolinkurl{justin.pargeter@nyu.edu}}}  %  \\
%             \emph{Present address:} of F. Author  %  if needed
    \and
        Megan Beney Kilgore 3 \at
     3. Department of Anthropology, Emory University, Atlanta, GA,
USA \\
     \email{\href{mailto:megan.elizabeth.beney@emory.edu}{\nolinkurl{megan.elizabeth.beney@emory.edu}}}  %  \\
%             \emph{Present address:} of F. Author  %  if needed
    \and
        Cheng Liu 3 \at
     3. Department of Anthropology, Emory University, Atlanta, GA,
USA \\
     \email{\href{mailto:raylc1996@outlook.com}{\nolinkurl{raylc1996@outlook.com}}}  %  \\
%             \emph{Present address:} of F. Author  %  if needed
    \and
        Dietrich Stout 3 \at
     3. Department of Anthropology, Emory University, Atlanta, GA,
USA \\
     \email{\href{mailto:dwstout@emory.edu}{\nolinkurl{dwstout@emory.edu}}}  %  \\
%             \emph{Present address:} of F. Author  %  if needed
    \and
    }

\date{Received: date / Accepted: date}
% The correct dates will be entered by the editor


\maketitle

\begin{abstract}
Stone tools provide key evidence of human cognitive evolution but remain
difficult to interpret. Toolmaking skill-learning in particular has been
understudied even though: 1) the most salient cognitive demands of
toolmaking should occur during learning, and 2) variation in learning
aptitude would have provided the raw material for any past selection
acting on tool making ability. However, we actually know very little
about the cognitive prerequisites of learning under different
information transmission conditions that may have prevailed during the
Paleolithic. This paper presents results from a pilot experimental study
to trial new experimental methods for studying the effect of learning
conditions and individual differences on Oldowan flake-tool making skill
acquisition. We trained 23 participants for 2 hours to make stone flakes
under two different instructional conditions (observation only
vs.~direct active teaching) employing appropriate raw materials,
practice time, and real human interaction. Participant performance was
evaluated through analysis of the stone artifacts produced. Performance
was compared both across experimental groups and with respect to
individual participant differences in grip strength, motor accuracy, and
cognitive function measured for the study. Our results show aptitude to
be associated with fluid intelligence in a verbally instructed group and
with a tendency to use social information in an observation-only group.
These results have implications for debates surrounding the cumulative
nature of human culture, the relative contributions of knowledge and
know-how for stone tool making, and the role of evolved psychological
mechanisms in ``high fidelity'' transmission of information,
particularly through imitation and teaching.
\\
\keywords{
        Oldowan \and
        Stone toolmaking \and
        Social learning \and
        Individual variation \and
        Cognitive aptitudes \and
        Motor skills \and
    }


\end{abstract}


\def\spacingset#1{\renewcommand{\baselinestretch}%
{#1}\small\normalsize} \spacingset{1}


\hypertarget{intro}{%
\section{Introduction}\label{intro}}

Stone tools have long been seen as a key source of evidence for
understanding human behavioral and cognitive evolution (Darwin 1871;
Oakley 1949; Washburn 1960). Pathbreaking attempts to infer specific
cognitive capacities from this evidence largely focused on the basic
requirements of tool production (Isaac 1976; Wynn 1979; Gowlett 1984;
Wynn and Coolidge 2004). More recently, increasing attention has been
directed to the processes and demands of stone tool making skill
acquisition (Roux, Bril, and Dietrich 1995; Stout 2002; Stout et al.
2005; Geribàs, Mosquera, and Vergès 2010; Nonaka, Bril, and Rein 2010;
Stout et al. 2011; Putt, Woods, and Franciscus 2014; Hecht, Gutman,
Khreisheh, et al. 2015; Duke and Pargeter 2015; Morgan et al. 2015;
Stout and Khreisheh 2015; Lombao, Guardiola, and Mosquera 2017; Putt et
al. 2017; Cataldo, Migliano, and Vinicius 2018; Putt, Wijeakumar, and
Spencer 2019; Pargeter, Khreisheh, and Stout 2019; Pargeter et al.
2020). This is motivated by the expectation that the most salient
cognitive demands of tool making should occur during learning rather
than routine expert performance (Stout and Khreisheh 2015) and by
interest in the relevance of different social learning mechanisms such
as imitation (Rein, Nonaka, and Bril 2014; Stout et al. 2019), emulation
(Tehrani and Riede 2008; Wilkins 2018), and language (Ohnuma, Aoki, and
Akazawa 1997; Putt, Woods, and Franciscus 2014; Morgan et al. 2015;
Lombao, Guardiola, and Mosquera 2017; Putt et al. 2017; Cataldo,
Migliano, and Vinicius 2018) to the reproduction of Paleolithic
technologies.

Studies investigating these questions have used a range of different
experimental designs (e.g., varying technological goals/instructions,
training times, raw materials, live vs.~recorded instruction,
lithic/skill assessment metrics, pseudo-knapping tasks etc.) and reached
disparate conclusions regarding the neurocognitive and social
foundations of skill acquisition. It is plausible that these discordant
results reflect actual diversity in how humans acquire and master stone
tool making skills. However, this failure of results to generalize
across artificial experimental manipulations
(\href{cf.\%20Yarkoni\%202020}{cf}. Yarkoni 2020) also raises doubts
regarding the external validity (Eren et al. 2016) of conclusions with
respect to real-world Paleolithic learning contexts. To address this, we
conducted an exploratory study that draws on lessons from previous
research in an attempt to balance the pragmatic and theoretical
tradeoffs inherent in experimental studies of stone knapping skill
acquisition (Pargeter, Khreisheh, and Stout 2019; Stout and Khreisheh
2015).

Learning real-world skills like stone knapping is highly demanding of
time and materials and difficult to control experimentally without
sacrificing generalizability to real world conditions. Prior efforts
have attempted to navigate these challenges by using various
combinations of 1) inauthentic raw materials that are less expensive,
easier to standardize, and/or easier to knap, 2) video-recorded
instruction that is uniform across participants and less demanding of
experimenter time, 3) short learning periods, 4) small sample sizes, and
5) single learning conditions. The difficulty of interpreting results
from this growing literature led Stout and Khreisheh (2015: 870,
emphasis original) to call for ``studies with sufficient sample sizes to
manipulate learning conditions (e.g.~instruction, motivation) and assess
individual variation (e.g.~performance, psychometrics, neuroanatomy)
that \emph{also} have realistic learning periods.'' The current study
attempts to strike a viable balance between these demands by
investigating early-stage learning of a relatively simple technology
(least effort, ``Oldowan,'' flake production (Reti 2016; Shea 2016)
under two instructional conditions while collecting data on individual
differences in strength, coordination, cognition, social learning,
self-control, and task engagement. Unlike any previous study, this
allows us to address the likelihood that group effects of training
conditions might be impacted by interactions with individual participant
differences in aptitude, motivation, or learning style.

We focus on early stage learning because it has been found to be
relatively rapid, variable across individuals, and predictive of later
outcomes (Pargeter, Khreisheh, and Stout 2019; Stout and Khreisheh 2015;
Putt, Wijeakumar, and Spencer 2019), and thus provides a reasonable
expectation of generating meaningful data on skill and learning
variation while minimizing training costs. Moreover, understanding the
minimum training times necessary to detect changes in tool making skill
will help archaeologists design more realistic and cost-effective
experiments. To further manage costs, we limited our study to only two
learning conditions (observation only vs.~active teaching). This targets
a key controversy in human evolution, namely the origins of teaching and
language (Gärdenfors and Högberg 2017; Morgan et al. 2015), while
avoiding highly artificial manipulations of dubious relevance to
real-world Paleolithic learning. These choices allowed us to invest more
in other aspects of research design that we identified as theoretically
important, including measurement of individual differences in cognition
and behavior, inclusion of an in-person, fully interactive teaching
condition, and use of naturalistic raw materials. Sample size remained
small in this internally funded exploratory study but could easily be
scaled up at funding levels typical of pre- and post-doctoral research
grants in archaeology.

\hypertarget{individual-differences}{%
\subsection{\texorpdfstring{\textbf{Individual
Differences}}{Individual Differences}}\label{individual-differences}}

``\emph{The many slight differences\ldots{} being observed in the
individuals of the same species inhabiting the same confined locality,
may be called individual differences\ldots{} These individual
differences are of the highest importance to us, for they are often
inherited \ldots{} and they thus afford materials for natural selection
to act on and accumulate\ldots{}}'' (Darwin 1859, Chapter 2)

Individuals vary in aptitude and learning style for particular skills
(Jonassen and Grabowski 1993) but this has largely been ignored in
studies of knapping skill acquisition, which have instead focused on
group effects of different experimental conditions. There are good
pragmatic reasons for this, as individual difference studies typically
require larger sample sizes and additional data collection. However,
overlooking these distinctions is not ideal since individual differences
can provide valuable insight into the mechanisms, development, and
evolution of cognition and behavior (Boogert et al. 2018). In
particular, patterns of association between cognitive traits and
behavioral performance can be used to test hypotheses about the
cognitive demands of learning particular skills and the likely targets
of natural selection acting on aptitude. More prosaically, individual
differences can introduce an unexamined and uncontrolled source of
variation in group level results. This is especially true in the
relatively small ``samples of convenience'' typical of experimental
archaeology.

While testing hypotheses in evolutionary cognitive archaeology remains a
considerable challenge (Wynn 2017), investigation of individual
variation in modern research participants represents one promising
direction. For any particular behavior of archaeological interest, it is
expected that standing variation in modern populations should remain
relevant to normal variation in learning aptitude. The presence of trait
variation without impact on learning aptitude would provide strong
evidence against the plausibility of the proposed evolutionary
relationship. An absence of variation (i.e., past fixation and rigorous
developmental canalization) is not expected given the known variability
of human brains and cognition (Sherwood and Gómez-Robles 2017; Barrett
2020). Any confirmatory findings of trait-aptitude correspondence would
then have the testable implication that humans should be evolutionarily
derived along the same dimension (e.g. Hecht, Gutman, Bradley, et al.
2015).

To date, a small number of ``neuroarchaeological'' studies have reported
associations between individual knapping performance and brain structure
or physiological responses. Hecht et al. (2015) reported
training-related changes in white matter integrity (fractional
anisotropy {[}FA{]}) that correlated with individual differences in
practice time and striking accuracy change. The regional patterning of
FA changes also varied across individuals, with only those individuals
who displayed early increases in FA under the right ventral precentral
gyrus (premotor cortex involved in movement planning and guidance)
showing striking accuracy improvement over the training period. Putt et
al. (2019) similarly found that the proportion of flakes to shatter
produced by individuals during handaxe making correlated with dorsal
precentral gyrus (motor cortex) activation. Pargeter et al. (2020) used
a flake prediction paradigm (modeled after Nonaka, Bril, and Rein 2010)
to confirm that striking force and accuracy are important determinants
of handaxe-making success. These findings all point to the central role
of perceptual-motor systems (Stout and Chaminade 2007) and coordination
(Roux, Bril, and Dietrich 1995) in knapping skill acquisition. In
addition, Putt et al. (2019) also found successful flake production to
be associated with prefrontal (working memory/cognitive control)
activation and Stout et al. (2015) found that prefrontal activation
correlated with success at a strategic judgement (platform selection)
task which in turn was predictive of success at out-of-scanner handaxe
production. Such investigations are thus starting to chart out the more
specific contributions of different neural systems to particular aspects
of knapping skill acquisition. To date, however, the
cognitive/functional interpretation of systems identified in this manner
has largely relied on informal reverse inference (reasoning backward
from observed activations to inferred mental processes) from published
studies of other tasks that activated the same regions, an approach
which is widely regarded as problematic (Poldrack 2011).

Here we take a more direct, psychometric approach to measuring
individual differences in perceptual-motor coordination and cognition.
Psychometric instruments (e.g., tasks, questionnaires) are designed to
assess variation in cognitive traits and states, such as fluid
intelligence, working memory, attention, motivation, and personality,
that have been of theoretical interest to cognitive archaeologists
(e.g., Wynn and Coolidge 2016). It is thus surprising that they have
been almost entirely neglected in experimental studies of knapping
skill. In the only published example we are aware of, Pargeter et al.
(2019) reported significant effects of variation in planning and problem
solving (Tower of London test (Shallice, Broadbent, and Weiskrantz
1982)) and cognitive set shifting (Wisconsin Card Sort test (Grant and
Berg 1948)) on early stage handaxe learning. Of course, cognition is not
the only thing that can affect knapping performance. Flake prediction
experiments highlight the importance of regulating movement
speed/accuracy trade-offs (Nonaka, Bril, and Rein 2010; Pargeter et al.
2020) and studies of muscle recruitment (Marzke et al. 1998) and manual
pressure (Williams-Hatala et al. 2018; Alastair J. M. Key and Dunmore
2018) during knapping highlight basic strength requirements. Along these
lines, Key and Lycett (2019) found that individual differences in hand
size, shape, and especially grip strength were better predictors of
force loading during stone tool use than were attributes of the tools
themselves. However, we are unaware of any such studies of biometric
influences on variation in knapping success. Finally, the time and
effort demands of knapping skill acquisition suggest that differences in
personality (e.g., self-control and ``grit'' (Pargeter, Khreisheh, and
Stout 2019), motivation (Stout 2002), and social vs.~individual learning
strategies (Miu et al. 2020) might also affect learning outcomes. We are
again unaware of any previous studies that have assessed such effects.
In this study, we assessed all participants with a battery of tests
including grip strength, movement speed/accuracy, spatial working
memory, fluid intelligence, self-control, tendency to use social
information, and motivation/engagement with the tool making task. We
were particularly interested in the possibility that these variables
might not only impact learning generally, but might also have different
effects under different learning conditions.~

\hypertarget{teaching-language-and-tool-making}{%
\subsection{\texorpdfstring{\textbf{Teaching, Language, and Tool
Making}}{Teaching, Language, and Tool Making}}\label{teaching-language-and-tool-making}}

\emph{A creature that learns to make tools to a complex pre-existing
pattern\ldots must have the kind of abstracting mind that would be of
high selective value in facilitating the development of the ability to
communicate such skills by the necessary verbal acts.} (Montagu 1976:
267)

Possible links between tool making and language have been a subject of
speculation for nearly 150 years (Engles 2003, {[}1873{]}), if not
longer (Hewes 1993), although compelling empirical tests have remained
elusive. Over 25 years ago, Toth and Schick (1993) suggested that
experiments teaching modern participants to make stone tools in verbal
and non-verbal conditions could test the importance of language in the
social reproduction of Paleolithic technologies. Ohnuma et al. (1997)
were the first to implement this suggestion in a study of Levallois
flake production, followed by more recent studies of handaxe making
(Putt, Woods, and Franciscus 2014; Putt et al. 2017) and simple flake
production (Morgan et al. 2015; Cataldo, Migliano, and Vinicius 2018;
Lombao, Guardiola, and Mosquera 2017). This reflects recent interest in
the hypothesis that language might be an adaptation for teaching (e.g.,
Laland 2017; Stout and Chaminade 2012). Teaching and learning demands of
Paleolithic tool making would thus provide evidence of selective
contexts favoring language evolution (Stout 2010; Morgan et al. 2015;
Montagu 1976).

Toth and Schick (1993) were, however, careful to point out that extinct
hominid learning strategies and capacities might differ from modern
experimental participants. Even leaving aside potential species
differences in social learning (cf. Morgan et al. 2015; Stout et al.
2019), reliance on explicit verbal instruction varies widely across
modern human societies (e.g., Boyette and Hewlett 2017). The WEIRD
(Western, educated, industrialized, rich, democratic (Henrich, Heine,
and Norenzayan 2010)) teachers and learners typical of knapping
experiments arguably represent an extreme bias toward such instruction.
Simply instructing such participants not to speak during an experiment
(or to demonstrate but not gesture, etc. (Morgan et al. 2015)) is likely
to underestimate the efficacy of non-verbal teaching and learning in
cultural contexts where it is more common, let alone in a hypothetical
pre-linguistic hominid species.

Such concerns are exacerbated in experiments using pre-recorded
instructional videos or extremely short training periods. Video does not
allow the interactive teaching that is favored even in formal academic
knapping classes (e.g., Shea 2015) and is almost certainly typical of
traditional learning contexts (e.g., Stout 2002). It is not known how
video presentation affects the efficacy of teaching generally, or the
relative effectiveness of different forms of instruction. Going further,
some experiments have manipulated the presence/absence of verbal
instruction by presenting the same video with and without sound (Putt et
al. 2017) or the sound track without the video (Cataldo, Migliano, and
Vinicius 2018). While this provides experimental control, it does not
allow the instructor to adjust their multi-modal (Levinson and Holler
2014) communication strategies as they would naturally do, for example
through pointing and pantomime. To simply remove a communication channel
without allowing any such adaptation is highly artificial and risks
generating results that cannot be generalized beyond the specific
context of the experiment (Yarkoni 2020). Similarly, unnaturally short
training periods (e.g., 5-15 minutes (Morgan et al. 2015; Lombao,
Guardiola, and Mosquera 2017)) might misrepresent the relative efficacy
of different teaching strategies under more realistic conditions (Whiten
2015; Stout and Khreisheh 2015). Even the longest training times to date
(Stout and Khreisheh 2015; Pargeter, Khreisheh, and Stout 2019) have not
produced knapping skills comparable to relevant archaeological examples,
and were achieved by limiting sample size and using only one teaching
condition.

For these reasons, we sought to explore a middle path between
experimental expedience and realism by limiting our experiment to two
relatively naturalistic learning conditions and a moderate learning
period of two hours. As in previous experiments (Stout et al. 2011;
Hecht, Gutman, Khreisheh, et al. 2015; Pargeter, Khreisheh, and Stout
2019) the first condition was unrestricted, interactive instruction in
small groups, essentially reproducing the ``natural'' teaching/learning
context familiar (cf. Shea 2015) to our WEIRD instructor and student
participants. The second condition allowed observation only, with the
experimenter visible making flakes but not interacting in any way with
learners. This absence of teaching is again a familiar social context
for our participants and did not require any novel behaviors from the
instructor. It matches the ``imitation/emulation'' condition of Morgan
et al. (2015) although we make no assumptions regarding learning
mechanisms. We did not include a ``reverse engineering'' or ``end-state
emulation'' condition in which only finished products were visible. This
has been advocated as an important baseline or control condition (Whiten
2015) to distinguish observational from individual learning, but is not
likely to model any typical Paleolithic learning context nor to stand as
an adequate proxy for the cognition of hominid species with different
social learning capacities. There is no reason to assume neurocognitive
and behavioral processes of reverse-engineering problem solving in
modern humans (e.g., Allen, Smith, and Tenenbaum 2020) approximate the
social learning processes of hominids with more ape-like action
observation/imitation capacities (Hecht, Gutman, et al. 2013; Hecht,
Murphy, et al. 2013; Stout et al. 2019).

We selected a two-hour learning period for both pragmatic and
theoretical reasons. Pargeter et al. (Pargeter, Khreisheh, and Stout
2019) found that even \textasciitilde90 hours of fully interactive
instruction and practice was insufficient to achieve handaxe-making
skills comparable to the later Acheulean site of Boxgrove
(García-Medrano et al. 2019; Stout et al. 2014), and estimated actual
time to mastery as ranging from 121 to 441 hours for different
participants. However, they observed the greatest, fastest, and most
individually variable skill increases during the first 20 hours of
practice. In addition, initial performance was moderately correlated
with later achievement. This suggests that studying early-stage learning
may be a pragmatic alternative, especially for research investigating
individual differences in aptitude. Studies of simple flake production
similarly document large initial variation (Stout and Khreisheh 2015)
and rapid early progress (Stout and Khreisheh 2015; Putt, Wijeakumar,
and Spencer 2019; Stout and Semaw 2006). We designed the current study
to test the utility of studying learning and variation during the first
two hours of simple flaking instruction/practice, in hopes of finding a
viable compromise between experimental realism and cost

\hypertarget{raw-materials-and-knapping-skill}{%
\subsection{\texorpdfstring{\textbf{Raw materials and knapping
skill}}{Raw materials and knapping skill}}\label{raw-materials-and-knapping-skill}}

Lithic raw materials vary in size, shape, and fracture mechanical
properties that affect the difficulty of achieving different knapping
goals (Eren et al. 2014). Unfortunately, it can be difficult and/or
expensive to procure authentic raw materials. Experimental studies of
knapping skill have often used proxy materials such as flint (Morgan et
al. 2015; Nonaka, Bril, and Rein 2010; Cataldo, Migliano, and Vinicius
2018), limestone (Stout and Semaw 2006), porcelain (Khreisheh, Davies,
and Bradley 2013), or heat-treated chert (Putt, Woods, and Franciscus
2014; Putt et al. 2017; Putt, Wijeakumar, and Spencer 2019)to model
Oldowan and early Acheulean technologies executed in other materials. As
well as being more readily available, these proxies are generally easier
to knap. This has the benefit of reducing required practice time, but it
is unclear how it might affect learning demands more generally or the
efficacy of different learning conditions/strategies specifically.

To address this, some studies have attempted to more closely match
experimental and archaeological raw material types (Stout et al. 2011;
Duke and Pargeter 2015; Pargeter, Khreisheh, and Stout 2019). However,
raw materials vary across individual clasts within as well as between
types. This has led to interest in standardizing experimental core
morphology (Nonaka, Bril, and Rein 2010) and composition, even if this
means using artificial materials such as porcelain (Khreisheh, Davies,
and Bradley 2013), brick (Geribàs, Mosquera, and Vergès 2010; Lombao,
Guardiola, and Mosquera 2017), or foam blocks (Schillinger, Mesoudi, and
Lycett 2014). Such manipulations enhance experimental control and
internal validity (Eren et al. 2016) at the expense of external
generalizability to actual archaeological conditions. Specifically, they
allow more robust results from smaller samples but eliminate a core
element of real-world knapping skill: the ability to produce consistent
results from variable materials (Pelegrin 1990; Stout 2013). For
example, Pargeter et al. (2020) found that predicting specific flaking
outcomes on actual handaxe preforms was both more difficult and less
technologically important than expected from previous work with
standardized, frustum-shaped cores (Nonaka, Bril, and Rein 2010). The
alternative to control is to incorporate raw material size, shape, and
composition as experimental variables (e.g., Stout et al. 2019). This
allows consideration of raw material selection and response to variation
as aspects of skill but correspondingly increases the sample sizes
required to identify patterning. In considering these issues, we again
chose to explore a middle path between pragmatism and realism by
employing commercially purchased basalt similar to that known from East
African Oldowan sites, allowing clast size and shape to vary within set
limits, and selecting the particular clasts provided to each participant
to approximate the same distribution.

\hfill\break

\hypertarget{materials-and-methods}{%
\section{\texorpdfstring{\textbf{Materials and
Methods}}{Materials and Methods}}\label{materials-and-methods}}

This research was approved by the Emory Institutional Review Board
(IRB00113024). All participants provided written informed consent and
completed a video release form
(\url{https://databrary.org/support/irb/release-template.html}).

\hypertarget{participants}{%
\subsection{\texorpdfstring{\textbf{Participants}}{Participants}}\label{participants}}

Twenty-four adult participants with no prior stone knapping experience
were recruited from the Emory community using paper fliers and e-mail
listserv advertisements. We were unable to replace one participant who
failed to attend their scheduled session, resulting in a total sample of
23. Eleven participants (6 female, 5 male) completed the Untaught
condition and 12 (8 female, 4 male) completed the Taught condition.

\hypertarget{study-visit}{%
\subsection{\texorpdfstring{\textbf{Study
Visit}}{Study Visit}}\label{study-visit}}

Participants were asked to visit the Paleolithic Technology Lab at Emory
University to complete one three-hour session. Participants were
scheduled to attend in six groups of four, however one of these groups
had only three participants due to a no-show on the day of the
experiment. Each visit began with the collection of individual
differences measures, which took approximately one hour. After that,
participants undertook 105 minutes (two hours minus a 15-minute break
after 1 hour) of stone tool making practice. This session was
video-recorded, and all lithic products were collected. After the tool
making task, participants completed an ``exit questionnaire'' comprising
the Intrinsic Motivation Inventory (see below).

Participants were compensated for their time with a \$30 gift card. They
also had the opportunity to earn a performance bonus of \$5, \$10, \$15
or \$20 on the gift card. They were told that this bonus would depend on
``how well they did'' on the last core of their practice session. The
actual performance measure was not specified, but in order to allow on
the spot payment a simple measure of the percentage of starting weight
removed from the final core was used such that: \textgreater{} 30\%
earned \$5, \textgreater{} 40\% earned \$10, \textgreater{} 50\% earned
\$15, \textgreater{} 75\% earned \$20.

\hfill\break

\hypertarget{individual-difference-measures}{%
\subsection{\texorpdfstring{\textbf{Individual Difference
Measures}}{Individual Difference Measures}}\label{individual-difference-measures}}

We used five individual difference measures for this study:

\begin{enumerate}
\def\labelenumi{\arabic{enumi})}
\item
  Grip strength was measured in kilograms using an electronic hand
  dynamometer (Camry EH101). Strength was measured twice and the higher
  value recorded. Grip strength is a simple measure that is well
  correlated with overall muscular strength (Wind et al. 2010) and a
  range of other health and fitness measures (Sasaki et al. 2007). It is
  hypothesized to be relevant to generating kinetic energy for fracture
  initiation (Nonaka, Bril, and Rein 2010) as well as control and
  support of the hammerstone (Williams-Hatala et al. 2018) and core
  (Faisal et al. 2010; Alastair J. M. Key and Dunmore 2015).
\item
  Motor accuracy was assessed using a ``Fitts Law'' reciprocal tapping
  task. Fitts Law describes the trade-off between speed and accuracy in
  human movement, classically measured by tapping back and forth between
  two targets of varying size and spacing (Fitts 1954). Archaeologists
  have proposed (Stout 2002; Pargeter et al. 2020) that management of
  this trade-off is critical to the accurate application of appropriate
  force seen in skilled knapping (Nonaka, Bril, and Rein 2010; Roux,
  Bril, and Dietrich 1995). We implemented this test on a Surface Pro
  tablet running free software (FittsStudy Version 4.2.8, default
  settings) developed by the Accessible Computing Experiences lab (Jacob
  O. Wobbrock, director) at the University of Washington
  (\url{depts.washington.edu/acelab/proj/fittsstudy/index.html}).
  Participants use a touchscreen pen to tap between ribbons on the
  screen, with average movement time as the performance metric.
\item
  Visuospatial working memory is the capacity to ``hold in mind,'' which
  researchers have hypothesized to be important in stone toolmaking
  performance (Coolidge and Wynn 2005). It also might support a learning
  process known as `chunking,' in which multiple items or operations are
  combined into summary chunks stored in long term memory, that is
  thought to be important in the acquisition of knapping and other
  skills (Pargeter, Khreisheh, and Stout 2019). We measured visuospatial
  working memory using a free n-back task
  (\href{https://wmp.education.uci.edu/software/}{wmp.education.uci.edu/software/})
  developed by the Working Memory and Plasticity Laboratory at the
  University of California, Irvine (Susanne Jaeggi, PI) and implemented
  in E-Prime software on a desktop computer. In this task, participants
  are asked to remember the position of blue squares presented
  sequentially on the screen and touch a key when the current position
  matches that 1, 2, 3\ldots n iterations back. Progression to blocks
  with increasing values of n is contingent on exceeding a threshold
  success rate. Performance was measured as the highest n achieved.
\item
  Fluid intelligence (Cattell 1963) refers to the capacity to engage in
  abstract reasoning and problem solving in a way that is minimally
  dependent on prior experience. It complements ``crystallized
  intelligence'' (the ability to apply learned procedures and knowledge)
  as one of the two factors (gf, gc) comprising so-called ``general
  intelligence'' (g). Fluid intelligence is closely related to the
  executive control of attention and manipulation of information held in
  working memory (Engle 2018)(Engle 2018). It is hypothesized to support
  technological innovation (Coolidge and Wynn 2005) and/or the
  intentional learning of new skills (Unsworth and Engle 2005; Stout and
  Khreisheh 2015). We measured fluid intelligence using the short
  version (Bilker et al. 2012) of the classic Raven Progressive Matrices
  task, which requires participants to complete increasingly difficult
  pattern matching questions.~~
\item
  The use of social information for learning and decision making varies
  across individuals and societies (Molleman, Kurvers, and van den Bos
  2019). Such variation is a key topic for understanding social learning
  and cultural evolutionary processes (Kendal et al. 2018; Heyes 2018;
  Miu et al. 2020) and represents a potential confound for assessing
  experimental effects of different social learning conditions. We
  measured participants' tendency to rely on social information
  vs.~their own insights using the Berlin Estimate AdjuStment Task
  (BEAST) developed by Molleman et al. (2019). In this task,
  participants are present with large arrays of items on a screen and
  asked to estimate the number present. They are then provided with
  another person's estimate and allowed to provide a second estimate.
  The participants' average adjustment between first and second
  estimates provides a measure of their propensity to rely on social
  information.
\end{enumerate}

\hfill\break

\hypertarget{stone-tool-making}{%
\subsection{\texorpdfstring{\textbf{Stone Tool
Making}}{Stone Tool Making}}\label{stone-tool-making}}

After individual difference testing, participants engaged in a 2-hour
stone tool making session, with a 15-minute break after 1 hour.
Participants were instructed not to seek out additional training or
information on stone tool making (i.e., via the internet) during these
breaks. Each group of participants was randomly assigned to one of two
experimental conditions: no teaching or teaching. In both conditions,
participants were first given an opportunity to inspect and handle
examples (\textbf{Figure}) of the kind of stone tools (flakes) they are
being asked to produce. They were told that their objective was to
produce as many flakes as possible from the materials provided. This
meant that even the untaught condition included some minimal instruction
(being told the objective) , however this was considered to be
unavoidable without creating a much more elaborate and naturalistic
context in which participants would develop their own technological
goals. Such a design would also be expected to increase behavioral
variability, demanding correspondingly larger samples of participants to
identify patterns and making direct comparisons with the taught
condition.

\hypertarget{raw-materials}{%
\subsubsection{\texorpdfstring{\textbf{Raw
Materials}}{Raw Materials}}\label{raw-materials}}

Each participant was provided with 9 cores for use over the 2-hour
experiment. These cores were produced from larger chunks of a
fine-grained basalt purchased from neolithics.com by fracturing them
with a sledgehammer. This produced irregular, angular chunks for use in
the experiment, weighing between 459g - 1876g (mean = 975g). All cores
were weighed, measured (Length, Width, Thickness), and painted white so
that new fracture surfaces could be discriminated from those created
during production. Cores were sorted by shape and weight and then
distributed evenly to each participant. As a result, there were no
significant difference across participants in the mean weight (ANOVA, df
= 22, F=0.3, p = 0.9; Levene test of homogeneity of variance = 1.04,
df1=22, df2 = 184, p = 0.4) or shape (Length x Width/Thickness: ANOVA,
df = 22, F=0.4, p = 0.9; Levene statistic = .6, df1=22, df2 = 184, p =
0.9) of cores provided. This was also true comparing the two
experimental conditions (Taught vs.~Untaught mean weight = 1001g
vs.~956g, t = 1.24, df = 205, p = 0.2, Levene's Test F = 0.6, p = 0.4;
mean shape = 221.43 vs.~221.45, t = -0.003, df = 205, p = 0.9, Levene's
Test F = 3.8, p = 0.05). Participants were, however, allowed to choose
which cores to work on so that differences in the weight and shape of
cores actually used across participants and conditions could still
emerge as a result of selection bias.

Sixty pounds of 3-to-5 inch basalt ``Mexican Beach Pebbles'' were
purchased from a landscaping supply company for use as hammerstones in
the experiment. Of these, 90 were selected as suitable for use. These
weighed between 213g-1360g (mean = 425) and varied in elongation (L/W =
1.01 to 2.65) and relative thickness (LxW/T = 90.48 to 283.67).
Forty-five stones were placed in the middle of the knapping area (Figure
2) for participants to freely choose from during the experiment. Broken
hammerstones were replaced from the reserve to maintain a consistent
number and range of choices. Each hammerstone was numbered and
participants' choices were recorded along with the number of the core(s)
being worked on with a particular hammerstone.\\

\hypertarget{experimental-conditions}{%
\subsubsection{\texorpdfstring{\textbf{Experimental
Conditions}}{Experimental Conditions}}\label{experimental-conditions}}

In both conditions, three researchers were present to record activities
and collect materials. Participants were seated in a circle (Shea 2015)
and experiments were video recorded using two cameras (Figure 1).
Participants were free to select hammerstones from the common pile and
to work on any or all of their nine assigned cores in any order they
preferred. However, each core and all associated debitage were collected
before participants were allowed to start working on a new core, so it
was not possible to partially work and then return to a particular core
later. The order of cores used and associated hammerstones were recorded
for each participant during the experiment.

In the untaught condition, a researcher (DS) sat with the participants
and made stone tools but remained silent and made no effort to
facilitate learning (e.g., through gesture, modified performance, facial
expression, attention direction, or verbal instruction). Over the 2-hour
period, the researcher completely reduced four cores (one every
\textasciitilde30 minutes). Participants were not restricted from
talking to each other, as this would create an unnatural and potentially
stressful social context that might affect learning. Participants were
asked to avoid any form of communication about the tool making task
specifically, and they complied with this request. Participants in this
condition thus had the opportunity to observe tool making by an expert
and/or by other learners, should they choose to do so, but received no
intentional instruction.

In the Taught condition, there were no restrictions on participant
interaction and the researcher engaged in direct active teaching (Kline
2015) of tool-making techniques through verbal instruction,
demonstration, gesture, and shaping of behavior. The instructor has a
moderate level of experience teaching basic knapping skills to students
in undergraduate archaeology classes and to participants in previous
knapping research (e.g., Stout et al. 2011). The pedagogical strategy
employed was based on the instructor's own learning experiences and
theoretical interpretations (e.g., Pargeter et al. 2020), and focused on
coaching participants in effective body postures, movement patterns, and
grips as well as the assessment of viable core morphology.~ ~

\textbf{Lithic Analysis}

All finished cores were weighed and measured (L, W, T). Delta weight was
calculated as (Start weight-End weight)/Start weight. All detached
pieces (DPs) were collected and weighed. We did not sort DPs into types
(e.g., whole flakes, fragments) as this would have greatly increased
processing time and it is not clear that such distinctions add relevant
information regarding utility/desirability beyond that supplied by
metrics (Stout et al. 2019). All DPs larger than 40mm in maximum
dimension were photographed and measured. It is conventional in Early
Stone Age lithic analysis to employ a 20 mm cut-off. We selected a
higher threshold for both pragmatic (analysis time) and theoretical
reasons. Flake use experiments have shown that flakes weighing less than
5--10 g or with a surface area below 7--10 cm2 (Prasciunas 2007) or with
a maximum dimension \textless50-60 mm (Alastair J. M. Key and Lycett
2014) become markedly inefficient for basic cutting tasks. Similarly,
data from Oldowan replication experiments (Stout et al. 2019) show that
the utility index (flake cutting edge/flake mass1/3) * (1 - exp{[}-0.31
* (flake maximum dimension -- 1.81){]}) developed by Morgan et
al.~(2015) falls off rapidly below 40mm maximum dimension ( Mean Utility
\textless{} 40mm = 0.508; \textgreater=40mm = 0.946; t= 11.99, df = 707,
p \textless{} 0.000). By including weight in our cut-off criteria we
also avoid skewing the flake shape distribution by selectively retaining
long, thin pieces (i.e., MD \textgreater{} 40, weight \textless{} 5g)
while discarding rounder pieces of similar (or greater) weight and area.

For measurement, DP length was defined as the longest axis and width as
the maximum dimension orthogonal to length. Thickness was defined as the
maximum dimension orthogonal to the plane formed by L and W and was
measured using calipers. L, W, and plan-view area measurements were
taken from photographs captured using a Canon Rebel T3i fitted with a 60
mm macro lens and attached to a photographic stand with adjustable upper
and lower light fittings. The camera was positioned directly above the
flakes and kept at a constant height. DPs were positioned irrespective
of any technological features so that the longest axis was vertical, and
the wider end was placed toward the bottom of the photograph.

Photographs were post-processed using Equalight software to adjust for
lens and lighting falloff that result from bending light through a lens
and its aperture which can affect measurements taken from photographs.
Each image was shot with a scale that was then used to rectify the
photograph's pixel scale to a real-world measurement scale in Adobe
Photoshop. Images were converted to binary black and white format and
silhouettes of the tools were extracted in Adobe Photoshop. We then used
a custom ImageJ (Rueden et al. 2017) script (Pargeter, Khreisheh, and
Stout 2019) to measure DP length and take nine width measurements at
10\% increments of length starting at the base of each DP. We used the
built-in ImageJ tool to measure DP area. A ``Proportion Larger DPs'' was
calculated per core as the combined weight of all DPs \textgreater40mm
in maximum dimension and 5g in weight divided by the weight of all DPs.
Higher values show cores with proportionally more large DPs.

\hypertarget{statistical-analyses}{%
\subsection{\texorpdfstring{\textbf{Statistical
Analyses}}{Statistical Analyses}}\label{statistical-analyses}}

To evaluate the association between psychometric, motor-skill, and
training measures and technological outcomes, we adopted an
information-theoretic approach (Burnham and Anderson 2002).
Information-theoretic approaches provide methods for model selection
using all possible combinations of variables while avoiding problems
associated with significance-threshold stepwise selection. We used the
corrected Akaike information criterion (AICc) to rate each possible
combination of predictors on the balance between goodness of fit
(likelihood of the data given the model) and parsimony (number of
parameters). The AICc consists of the log likelihood (i.e., how well
does the model fit the data?) and a penalty term for the number of
parameters that must be estimated in the model (i.e., how parsimonious
is the model?), with a correction for small sample sizes (AICc converges
to the standard AIC at large samples). A lower AICc indicates a more
generalizable model and we used it to compare and rank various possible
models. Each analysis begins with a full model that includes all
predictors of interest. All possible combinations of predictors are then
fit, and the resulting models are ranked and weighted based on their
AICc. The ``best'' model is chosen because it has the lowest AICc score.

Continuous predictors were centered such that zero represents the sample
average, and units are standard deviations. The full model was fitted
with the lm function in R 3.2.3, and the glmulti package
(\textbf{Bartoń?}) was used for multi-modal selection and model
comparison.

\hfill\break

\hypertarget{results}{%
\section{\texorpdfstring{\textbf{Results}}{Results}}\label{results}}

Following a recent protocol to enhance the reproducibility and data
transparency of archaeological research (Marwick 2017), detailed results
of all analyses and assessments of the data structure are available in
our paper's supplementary materials and through Github
(\url{https://github.com/Raylc/PaST-pilot}). Here we limit discussion to
the major findings regarding flaking performance and individual
differences. The purpose of this section is to determine whether
training impacted subject flaking performance and if any of the
subject's individual psychometric and motor-skill aptitudes predict
flaking performance.

\hypertarget{principal-component-analyses}{%
\subsection{\texorpdfstring{\textbf{Principal Component
analyses}}{Principal Component analyses}}\label{principal-component-analyses}}

The following two sections outline factor analyses designed to summarize
our main study metrics tracking individual variation in flake sizes and
shapes and lithic performance measures.

\hypertarget{flake-size-and-shape}{%
\subsubsection{\texorpdfstring{\textbf{Flake size and
shape}}{Flake size and shape}}\label{flake-size-and-shape}}

To better understand the relationship between flake shape and
training/individual variation, we entered our nine flake linear plan
measurements along with maximum flake length and thickness into a
principal component analysis (PCA) from which summary coordinates were
extracted. Bartlett's Test of Sphericity was significant (\(\chi ^2\)
(10) =4480, p \textless{} .01) indicating that the set of variables are
adequately related for factor analysis.

The analysis yielded three factors explaining a total of 90\% of the
variance for the entire 11 measurement variable set \textbf{(Table}).
Factor 1 tracks flake size with higher scores indicating larger flakes
since all 11 measures load positively on this factor. Factor 2's
loadings track the increasing relationship between thickness, length,
and flake width. As factor 2 scores increase, flakes get thicker,
longer, and narrow, resembling irregular splinters. Factor 3 tracks the
relationship between flake proximal and distal width relative to
thickness. As factor 3 scores go up, flakes get thinner and narrower at
the distal ends and wider at the base. Factor 3 therefore tracks flakes
with a typical shape having a thin cross-section, wider base, and
narrower tip. We used these three flake shape coordinates to approximate
flake size and shape in the project's flake performance factor analysis.

\hypertarget{lithic-flaking-performance-measures}{%
\subsubsection{\texorpdfstring{\textbf{Lithic flaking performance
measures}}{Lithic flaking performance measures}}\label{lithic-flaking-performance-measures}}

To better understand the relationship between our various lithic
performance measurements and to reduce these data dimensionality, we
conducted a second principal component analysis examining the study's
seven lithic performance measures (mass of flakes relative to flaked
core mass, count of large flakes {[}\textgreater40mm and 5g{]}, core
delta mass, three flake shape factors, and the total number of cores
used). All of these measures except the total cores flaked were
summarized for each core. The Bartlett's Test of Sphericity was
significant (\(\chi ^2\) (6) =3950, p \textless{} .01) indicating that
the set of variables are at least adequately related for factor
analysis.

The analysis yielded two factors explaining a total of 52.3\% of the
variance for the entire set of variables. Factor 1 tracks flaking
quantity due to high positive loadings on mass flakes/flaked mass, core
delta mass, total cores used, and flake shape factor 2. Negative
loadings on flake factor 1 (flake size) and 3 (basal/tip width shape)
suggest that flaking quantity comes at the expense of producing lots of
small and thick stone splinters/chunks. This first factor explained 28\%
of the variance. Higher factor 1 values reflect higher flaking
quantities. The second factor covers 24\% of the sample variance. Factor
2 measures one's ability to carefully flake cores due to high positive
loadings on large flake count and lower or neutral loadings on mass
flakes/flaked mass, core delta mass, and total cores used. The resulting
flakes are larger, relatively thinner, and more typically shaped due to
high positive loadings on flake factor 1 (size) and flake factor 2
(relative thickness) contrasted with a high negative loading on flake
shape factor 3 (basal/tip width). Higher factor two values show
increased quality flaking performance.

\textbf{Figure} shows the covariance between our two flake performance
principal components. While there is no significant relationship between
the two factors, and the two study groups do not show significantly
different slopes (p = 0.3), there are never-the-less differences between
the two groups. Trained individuals are able to increase flake quality
while increasing flake quality, whereas untrained individuals increase
their flaking quantity at the expense of flake quality.

\hypertarget{do-trained-untrained-and-expert-knappers-perform-differently}{%
\subsection{\texorpdfstring{\textbf{Do trained, untrained, and expert
knappers perform
differently?}}{Do trained, untrained, and expert knappers perform differently?}}\label{do-trained-untrained-and-expert-knappers-perform-differently}}

Here we compare our flaking outcomes (flake size/shape and flaking
performance factors) between the trained and untrained groups to see if
training in any way impacted their flaking performance. We added a
performance comparison (individual flake and core knapping performance
metrics) between these two novice groups and our expert knapper to test
for differences at different points in the stone flaking skill
spectrum.~

\textbf{Table} summarizes the group level performance tests. The results
show no significant differences in flaking performance between the
trained and untrained groups measured by our two flaking performance
factors. Trained and untrained individuals overall performed equally as
well in terms of flaking quality and quantity. Three-way flake size and
shape comparisons between our expert knapper and the two novice groups
show significant differences in flake shape factor 2 (relative
thickness), but with a very low (\textless0.01) effect size. This
difference is driven by the expert's lower overall relative flake
thickness. Regardless of training, our experimental subjects produced
flakes that were on average the same size and shape as those of the
expert trainer. We did, however, find several significant differences in
the three-way comparisons of our individual flaking performance
measures. The expert knapper made significantly more large flakes
(effect size = 0.14), had a significantly higher core delta mass signal
than either of the novice groups (effect size = 0.26), and on average
had significantly smaller cores (effect size = 0.27) (\textbf{Figure}).
All three of these results show either medium or large effect sizes. In
all three comparisons, the trained group's data distributions tended
towards the expert sample (although they were not significantly
different from the untrained group) (\textbf{Figures-Core examples
too}). These results show that core reduction intensity and large flake
production track this experiment's greatest differences between expert
and novice performance.

\hypertarget{does-trainingpractice-time-impact-flaking-performance}{%
\subsection{\texorpdfstring{\textbf{Does training/practice time impact
flaking
performance?}}{Does training/practice time impact flaking performance?}}\label{does-trainingpractice-time-impact-flaking-performance}}

Here we use the relative order in which each subject flaked their cores
to test for changes in flaking performance across the 2hr experiment.
For these analyses, we calculated the relative percentage for each core
relative to the total cores knapped by each subject. These relative core
use-order percentages were then binned into 20 percent brackets for
core-order and group-level comparisons. Flaking outcomes were tracked
using the two flake performance factors (quality and quantity flaking).
We added the nodule starting mass to track whether training/practice
times impacted raw material selection.

\textbf{Table} shows no significant training effects across the two
flaking performance measures either as grouped data or between
individuals (\textbf{Figures)}. This result demonstrated that flaking
outcomes did not change dramatically across the study interval. The one
significant main training effect related to core starting mass (with a
strong main training effect size = 0.25). On average, core starting
masses increase as subjects flake more cores, showing that subjects
started with smaller nodules first that were smaller and easier to
hold/flake. As the experiment wore on, they were left with larger and
more challenging nodules. The small main effect of training condition is
driven by higher starting nodule masses in the untrained group at the
beginning, half way through, and at the end of the experiment.~

\hypertarget{do-individual-differences-in-motor-skill-and-psychometric-measures-predict-flaking-performance}{%
\subsection{\texorpdfstring{\textbf{Do individual differences in motor
skill and psychometric measures predict flaking
performance?}}{Do individual differences in motor skill and psychometric measures predict flaking performance?}}\label{do-individual-differences-in-motor-skill-and-psychometric-measures-predict-flaking-performance}}

One of the experiment's primary goals was to test if measures of
individual variation in motor skill and intelligence/motivation predict
success in stone flaking. To address this goal, we built two
multivariate models examining the relations between our various
psychometric measures, subject's motor skill scores, and our two lithic
performance factors (quantity flaking and quality flaking). These models
enabled us to determine which of the psychometric and motor skill
factors are better predictors of a participant's flaking performance in
the study.

We have already demonstrated that subjects selected progressively larger
nodules throughout the experiment. It is important now to understand
whether nodule variability had any impact on our flaking results.
Because starting nodule size (mass) and shape were strongly correlated
(F {[}1,157{]} = 186, p \textless{} 0.01, R2 = 0.54) we included nodule
mass as a covariate to control for any variance in flaking performance
that may be driven by nodule differences. Our two motor skill and
strength measures (grip strength and Fitt's performance scores) are also
strongly correlated (F {[}1,19{]} = 15, p \textless{} 0.01, R2 = 0.41).
However, these two measures track complementary components of
athleticism (strength vs.~speed/accuracy tradeoffs) and so we decided to
include both in the model selection process.

~

We considered all possible interactions between five individual
difference measures, core size, training condition, and the two lithic
flaking performance factors (quantity and quality flaking) (wherein each
subject provides one data point). Each model's continuous predictors
(highest n-back level, Raven's Progressive Matrix score, BEAST score,
starting nodule mass, Fitt's score, and grip strength) were centered
such that zero represents the sample average, and units are standard
deviations.~

\hypertarget{model-1-individual-differences-and-quantity-flaking}{%
\subsubsection{\texorpdfstring{\textbf{Model 1: Individual differences
and quantity
flaking}}{Model 1: Individual differences and quantity flaking}}\label{model-1-individual-differences-and-quantity-flaking}}

The first full model examined variance in the quantity of flaking
tracked by our first performance factor explaining increases in the
number of cores used, the degree of reduction on each core, and
large/relatively thick flake production. The full model was fitted with
the lm function in R 3.2.3, and we used the Glmulti package's automated
model selection algorithm to select the best performing model (lowest
AICc score) (see methods for further details on the multimodal selection
process). The complete model statement is as follows:

\emph{Quantity flaking \(\sim\) Training condition + Highest n-back
level + Raven's Progressive Matrix score + BEAST score + Fitt's score +
Grip strength}

From a candidate pool of 55893 possible multivariate models, the best
performing model returned an AICc value of 36 (Average AIC = 52). This
model comprised the following statement with three main and four
interaction effects:

\emph{Quantity flaking \(\sim\) Training condition + Highest n-back
level + BEAST score + Grip strength + Fitt's score\(\cdot\)Raven's
Progressive Matrix score + Training condition\(\cdot\)Highest n-back
level + Training condition\(\cdot\)BEAST score + Nodule mass (as
control)}

This model explains a statistically significant and substantial
proportion of variance in quantity flaking (R2 = 0.84, F (8, 12) = 8.2,
p \textless{} 0.01, adj. R2 = 0.74). A model evaluation comparing
training vs.~test data subsets shows no evidence for overfitting
(Difference in R2 between training and test models = 0.02). A model
residuals normality test shows no significant differences with the
normal distribution (p = 0.35) indicating that this relationship (as
required) is linear. A Breusch-Pagan test showed no evidence for
heteroskedasticity (whether variance for all observations in our data
set are the same) (BP = 8.2, df = 7, p = 0.3).

\hfill\break
\textbf{Table} presents this model's coefficients and summary outputs,
wherein baseline refers to the untrained condition with all continuous
predictors at the sample average. The parameter estimates for the
continuous predictors reflect the expected change in utility for 1
standard deviation change in the predictor variable. Significant
increases in quantity flaking were found for subjects with training
(Est. = 0.63 {[}0.15, 1.11{]}), higher n-back levels (Est. = 0.71
{[}0.29, 1.13{]}), grip strength (Est. = 0.66 {[}0.36, 0.95{]}), and
BEAST scores (Est. = 0.41 {[}0.04, 0.78{]}) regardless of starting
nodule sizes (\textbf{Figure}). This suggests that several independent
factors tracking the effects of training, social information use,
visuo-spatial working memory, and strength improve an individual's
ability to flake in greater quantities.

The model produced two significant interactions between training
condition and n-back level and BEAST scores. Subjects in the trained
group with higher n-back levels show lower quantity flaking scores (Est.
= -1.35 {[}-1.85, -0.84{]}) while subjects in the trained group with
higher BEAST scores also show lower quantity flaking scores (Est. =
-0.83 {[}-1.37, -0.28{]}) when all the other variables are held
constant. At first glance, these results appear to contrast with
expectations regarding the effects of visuo-spatial working memory and
social information use on success in technological tasks. However,
looking at the matter more holistically it is clear that untrained
subjects produce flaking quantities in a different way that relies on
visuo-spatial working memory rather than taught information. When
teaching is provided, being aware of one's surroundings (i.e.~copying
from other novices) can have a negative effect on flaking quantity. The
interaction between BEAST scores and training shows that social
information use acts to level the effects of an individual's propensity
to ``muscle'' through the flaking task (both groups converge on an
average quantity flaking score with higher BEAST scores). However, the
direction of this change is different depending on training with
untrained and socially inclined individuals increasing flaking
quantities in contrast to trained and socially inclined individuals
decreasing flaking quantities.

\hypertarget{model-2-individual-differences-and-quality-flaking}{%
\subsubsection{\texorpdfstring{\textbf{Model 2: Individual differences
and quality
flaking}}{Model 2: Individual differences and quality flaking}}\label{model-2-individual-differences-and-quality-flaking}}

Our second model examining variance in quality flaking follows the same
complete model statement we used for the quantity flaking with six
covariates. From the same candidate pool size of 55893 possible
multivariate models, the best performing model returned an AICc value of
32 (Average AIC = 41). This model comprised the following statement with
three main and four interaction effects:

\emph{Quality flaking \(\sim\) Highest n-back level + Fitt's score +
Grip strength + Fitt's score\(\cdot\)BEAST score + Grip
strength\(\cdot\)BEAST score + Grip strength\(\cdot\)Fitt's score +
Training condition\(\cdot\)Grip strength + Nodule mass (as control)}

This model explains a statistically significant and substantial
proportion of variance in careful flaking (R2 = 0.78, F (8, 12) = 5.5, p
\textless{} 0.01, adj. R2 = 0.64) in the absence of any main training
effects. A model evaluation comparing training vs.~test data subsets
shows some evidence for overfitting (Difference in R2 between training
and test models = 0.17), which is unfortunately unavoidable with our
small sample sizes. A model residuals normality test shows no
significant differences with the normal distribution (p = 0.13)
indicating that this relationship is linear. A Breusch-Pagan test showed
no evidence for heteroskedasticity (BP = 3.4, df = 8, p = 0.9).

\hfill\break
\textbf{Table} presents this model's coefficients and summary outputs
following the same data format as \textbf{Table}. The results show three
significant main effects between our measures of motor-skill, strength,
visuo-spatial working memory, and quality flaking. A one unit increase
in Fitt's score (Est. = -0.7 {[}-1.08, -0.31{]}), grip strength (Est. =
-0.9 {[}-1.32, -0.53{]}), visuo-spatial working memory (n-back level)
(Est. = -0.3 {[}-0.58, -0.07{]}), and nodule starting mass (Est. = -0.3
{[}-0.58, -0.04{]}) results in decreased flaking quality regardless of
training. Collectively, these results show that quality flaking comes at
a cost in terms of a subject's motor abilities and that quality flaking
decreases as subjects are faced with flaking larger and more challenging
nodules.

The quality flaking model shows two significant interaction effects
involving grip strength, BEAST scores, and training. The interaction
between grip strength and BEAST scores has a significant positive effect
on careful flaking (Est. = 0.5 {[}0.22, 0.86{]}). \textbf{Figure}
illustrates this interaction whereby a subject's propensity to use
social information mitigates against individual differences in grip
strength. Strong subjects with a low propensity to use social
information tend to perform worst in terms of quality flaking. This
result makes sense if one considers that our quantity flaking model
showed how social information equalizes performance levels in trained
and untrained subjects.~\\
~\\
The second interaction involves grip strength and training. A one unit
increase in this interaction results in a significant positive effect on
careful flaking (Est. = 1 {[}0.51, 1.47{]}), but only in the untrained
group. \textbf{Figure} illustrates this interaction showing how training
helps even out flaking performance differences between stronger and
weaker individuals. Stronger individuals perform worse in terms of
quality flaking when they are untrained. Trained subjects, regardless of
grip strength, perform equally well in terms of quality flaking.

\hypertarget{discussion}{%
\section{\texorpdfstring{\textbf{Discussion}}{Discussion}}\label{discussion}}

we have little basis other than personal experience and/or tradition
(Callahan 1979; Whittaker 1994; Shea 2015) and theoretical speculation
(Stout 2013; Whiten 2015) from which to assess what pedagogical
techniques are most effective even in WEIRD contexts. For example, no
study to date has considered how variation in teacher skill (Shea 2015)
or social relationship to participants might impact learning under
different conditions. To properly address these questions would require
a major research program, including both cross-cultural comparative
studies (Barrett 2020) (Barrett 2020) and more naturalistic study
designs. While costly, such research would produce results of broad
relevance to anthropologists, biologists, psychologists, and
sociologists interested in teaching and learning, well beyond any
particular implications for language evolution.

\hypertarget{conclusions}{%
\section{\texorpdfstring{\textbf{Conclusions}}{Conclusions}}\label{conclusions}}

\hypertarget{acknowledgments}{%
\section{\texorpdfstring{\textbf{Acknowledgments}}{Acknowledgments}}\label{acknowledgments}}

\hypertarget{references}{%
\section*{\texorpdfstring{\textbf{References}}{References}}\label{references}}
\addcontentsline{toc}{section}{\textbf{References}}

\hypertarget{refs}{}
\begin{CSLReferences}{1}{0}
\leavevmode\hypertarget{ref-allen2020}{}%
Allen, Kelsey R., Kevin A. Smith, and Joshua B. Tenenbaum. 2020.
{``Rapid Trial-and-Error Learning with Simulation Supports Flexible Tool
Use and Physical Reasoning.''} \emph{Proceedings of the National Academy
of Sciences} 117 (47): 29302--10.
\url{https://doi.org/10.1073/pnas.1912341117}.

\leavevmode\hypertarget{ref-barrett2020}{}%
Barrett, H. Clark. 2020. {``Towards a Cognitive Science of the Human:
Cross-Cultural Approaches and Their Urgency.''} \emph{Trends in
Cognitive Sciences} 24 (8): 620--38.
\url{https://doi.org/10.1016/j.tics.2020.05.007}.

\leavevmode\hypertarget{ref-bilker2012}{}%
Bilker, Warren B., John A. Hansen, Colleen M. Brensinger, Jan Richard,
Raquel E. Gur, and Ruben C. Gur. 2012. {``Development of Abbreviated
Nine-Item Forms of the Raven{'}s Standard Progressive Matrices Test.''}
\emph{Assessment} 19 (3): 354--69.
\url{https://doi.org/10.1177/1073191112446655}.

\leavevmode\hypertarget{ref-boogert2018}{}%
Boogert, Neeltje J., Joah R. Madden, Julie Morand-Ferron, and Alex
Thornton. 2018. {``Measuring and Understanding Individual Differences in
Cognition.''} \emph{Philosophical Transactions of the Royal Society B:
Biological Sciences} 373 (1756): 20170280.
\url{https://doi.org/10.1098/rstb.2017.0280}.

\leavevmode\hypertarget{ref-boyette2017}{}%
Boyette, Adam H., and Barry S. Hewlett. 2017. {``Autonomy, Equality, and
Teaching Among Aka Foragers and Ngandu Farmers of the Congo Basin.''}
\emph{Human Nature} 28 (3): 289--322.
\url{https://doi.org/10.1007/s12110-017-9294-y}.

\leavevmode\hypertarget{ref-burnham2002}{}%
Burnham, Kenneth P., and David R. Anderson. 2002. \emph{Model Selection
and Multimodel Inference: A Practical Information-Theoretic Approach}.
2nd ed. New York: Springer-Verlag. \url{https://doi.org/10.1007/b97636}.

\leavevmode\hypertarget{ref-callahan1979}{}%
Callahan, Errett. 1979. {``THE BASICS OF BIFACE KNAPPING IN THE EASTERN
FLUTED POINT TRADITION: A MANUAL FOR FLINTKNAPPERS AND LITHIC
ANALYSTS.''} \emph{Archaeology of Eastern North America} 7 (1): 1--180.
\url{https://www.jstor.org/stable/40914177}.

\leavevmode\hypertarget{ref-cataldo2018}{}%
Cataldo, Dana Michelle, Andrea Bamberg Migliano, and Lucio Vinicius.
2018. {``Speech, Stone Tool-Making and the Evolution of Language.''}
\emph{PLOS ONE} 13 (1): e0191071.
\url{https://doi.org/10.1371/journal.pone.0191071}.

\leavevmode\hypertarget{ref-cattell1963}{}%
Cattell, Raymond B. 1963. {``Theory of Fluid and Crystallized
Intelligence: A Critical Experiment.''} \emph{Journal of Educational
Psychology} 54 (1): 1--22. \url{https://doi.org/10.1037/h0046743}.

\leavevmode\hypertarget{ref-coolidge2005}{}%
Coolidge, Frederick L., and Thomas Wynn. 2005. {``Working Memory, Its
Executive Functions, and the Emergence of Modern Thinking.''}
\emph{Cambridge Archaeological Journal} 15 (1): 5--26.
\url{https://doi.org/10.1017/S0959774305000016}.

\leavevmode\hypertarget{ref-darwin1859}{}%
Darwin, Charles. 1859. \emph{On the Origin of Species by Means of
Natural Selection, or, The Preservation of Favoured Races in the
Struggle for Life}. 1st ed. London: John Murray.

\leavevmode\hypertarget{ref-darwin1871}{}%
---------. 1871. \emph{The Descent of Man, and Selection in Relation to
Sex}. 1st ed. London: John Murray.

\leavevmode\hypertarget{ref-duke2015}{}%
Duke, Hilary, and Justin Pargeter. 2015. {``Weaving Simple Solutions to
Complex Problems: An Experimental Study of Skill in Bipolar
Cobble-Splitting.''} \emph{Lithic Technology} 40 (4): 349--65.
\url{https://doi.org/10.1179/2051618515Y.0000000016}.

\leavevmode\hypertarget{ref-engle2018}{}%
Engle, Randall W. 2018. {``Working Memory and Executive Attention: A
Revisit.''} \emph{Perspectives on Psychological Science} 13 (2):
190--93. \url{https://doi.org/10.1177/1745691617720478}.

\leavevmode\hypertarget{ref-engles2003}{}%
Engles, Friedrich. 2003. {``The Part Played by Labour in the Transition
from Ape to Man.''} In, edited by Robert C. Scharff and Val Dusek,
71--77. London: Blackwell.

\leavevmode\hypertarget{ref-eren2016}{}%
Eren, Metin I., Stephen J. Lycett, Robert J. Patten, Briggs Buchanan,
Justin Pargeter, and Michael J. O'Brien. 2016. {``Test, Model, and
Method Validation: The Role of Experimental Stone Artifact Replication
in Hypothesis-Driven Archaeology.''} \emph{Ethnoarchaeology: Journal of
Archaeological, Ethnographic and Experimental Studies} 8 (2): 103--36.
\url{https://doi.org/10.1080/19442890.2016.1213972}.

\leavevmode\hypertarget{ref-eren2014}{}%
Eren, Metin I., Christopher I. Roos, Brett A. Story, Noreen von
Cramon-Taubadel, and Stephen J. Lycett. 2014. {``The Role of Raw
Material Differences in Stone Tool Shape Variation: An Experimental
Assessment.''} \emph{Journal of Archaeological Science} 49: 472--87.
\url{https://doi.org/10.1016/j.jas.2014.05.034}.

\leavevmode\hypertarget{ref-faisal2010}{}%
Faisal, Aldo, Dietrich Stout, Jan Apel, and Bruce Bradley. 2010. {``The
Manipulative Complexity of Lower Paleolithic Stone Toolmaking.''}
\emph{PLOS ONE} 5 (11): e13718.
\url{https://doi.org/10.1371/journal.pone.0013718}.

\leavevmode\hypertarget{ref-fitts1954}{}%
Fitts, Paul M. 1954. {``The Information Capacity of the Human Motor
System in Controlling the Amplitude of Movement.''} \emph{Journal of
Experimental Psychology} 47 (6): 381--91.
\url{https://doi.org/10.1037/h0055392}.

\leavevmode\hypertarget{ref-garcuxeda-medrano2019}{}%
García-Medrano, Paula, Andreu Ollé, Nick Ashton, and Mark B. Roberts.
2019. {``The Mental Template in Handaxe Manufacture: New Insights into
Acheulean Lithic Technological Behavior at Boxgrove, Sussex, UK.''}
\emph{Journal of Archaeological Method and Theory} 26 (1): 396--422.
\url{https://doi.org/10.1007/s10816-018-9376-0}.

\leavevmode\hypertarget{ref-guxe4rdenfors2017}{}%
Gärdenfors, Peter, and Anders Högberg. 2017. {``The Archaeology of
Teaching and the Evolution of Homo Docens.''} \emph{Current
Anthropology} 58 (2): 188--208. \url{https://doi.org/10.1086/691178}.

\leavevmode\hypertarget{ref-geribuxe0s2010}{}%
Geribàs, Núria, Marina Mosquera, and Josep Maria Vergès. 2010. {``What
Novice Knappers Have to Learn to Become Expert Stone Toolmakers.''}
\emph{Journal of Archaeological Science} 37 (11): 2857--70.
\url{https://doi.org/10.1016/j.jas.2010.06.026}.

\leavevmode\hypertarget{ref-gowlett1984}{}%
Gowlett, John A. J. 1984. {``Mental Abilities of Early Man: A Look at
Some Hard Evidence.''} \emph{Higher Education Quarterly} 38 (3):
199--220. \url{https://doi.org/10.1111/j.1468-2273.1984.tb01387.x}.

\leavevmode\hypertarget{ref-grant1948}{}%
Grant, David A., and Esta Berg. 1948. {``A Behavioral Analysis of Degree
of Reinforcement and Ease of Shifting to New Responses in a Weigl-Type
Card-Sorting Problem.''} \emph{Journal of Experimental Psychology} 38
(4): 404--11. \url{https://doi.org/10.1037/h0059831}.

\leavevmode\hypertarget{ref-hecht2015b}{}%
Hecht, Erin E., David A. Gutman, Bruce A. Bradley, Todd M. Preuss, and
Dietrich Stout. 2015. {``Virtual dissection and comparative connectivity
of the superior longitudinal fasciculus in chimpanzees and humans.''}
\emph{NeuroImage} 108 (March): 124--37.
\url{https://doi.org/10.1016/j.neuroimage.2014.12.039}.

\leavevmode\hypertarget{ref-hecht2015a}{}%
Hecht, Erin E., David. A. Gutman, Nada Khreisheh, S. V. Taylor, J.
Kilner, A. A. Faisal, Bruce A. Bradley, T. Chaminade, and D. Stout.
2015. {``Acquisition of Paleolithic toolmaking abilities involves
structural remodeling to inferior frontoparietal regions.''} \emph{Brain
Structure \& Function} 220 (4): 2315--31.
\url{https://doi.org/10.1007/s00429-014-0789-6}.

\leavevmode\hypertarget{ref-hecht2013a}{}%
Hecht, Erin E., David A. Gutman, Todd M. Preuss, Mar M. Sanchez, Lisa A.
Parr, and James K. Rilling. 2013. {``Process Versus Product in Social
Learning: Comparative Diffusion Tensor Imaging of Neural Systems for
Action Execution{{}}observation Matching in Macaques, Chimpanzees, and
Humans.''} \emph{Cerebral Cortex} 23 (5): 1014--24.
\url{https://doi.org/10.1093/cercor/bhs097}.

\leavevmode\hypertarget{ref-hecht2013b}{}%
Hecht, Erin E., Lauren E. Murphy, David A. Gutman, John R. Votaw, David
M. Schuster, Todd M. Preuss, Guy A. Orban, Dietrich Stout, and Lisa A.
Parr. 2013. {``Differences in Neural Activation for Object-Directed
Grasping in Chimpanzees and Humans.''} \emph{The Journal of
Neuroscience} 33 (35): 14117--34.
\url{https://doi.org/10.1523/JNEUROSCI.2172-13.2013}.

\leavevmode\hypertarget{ref-henrich2010}{}%
Henrich, Joseph, Steven J. Heine, and Ara Norenzayan. 2010. {``Most
People Are Not WEIRD.''} \emph{Nature} 466 (7302): 29--29.
\url{https://doi.org/10.1038/466029a}.

\leavevmode\hypertarget{ref-hewes1993}{}%
Hewes, Gordon W. 1993. {``A History of Speculation on the Relation
Between Tools and Language.''} In, edited by Kathleen R. Gibson and Tim
Ingold, 20--31. Cambridge: Cambridge University Press.

\leavevmode\hypertarget{ref-heyes2018}{}%
Heyes, Cecilia. 2018. {``Enquire Within: Cultural Evolution and
Cognitive Science.''} \emph{Philosophical Transactions of the Royal
Society B: Biological Sciences} 373 (1743): 20170051.
\url{https://doi.org/10.1098/rstb.2017.0051}.

\leavevmode\hypertarget{ref-isaac1976}{}%
Isaac, Glynn L. 1976. {``Stages of Cultural Elaboration in the
Pleistocene: Possible Archaeological Indicators of the Development of
Language Capabilities.''} \emph{Annals of the New York Academy of
Sciences} 280 (1): 275--88.
\url{https://doi.org/10.1111/j.1749-6632.1976.tb25494.x}.

\leavevmode\hypertarget{ref-jonassen1993}{}%
Jonassen, David H., and Barbara L. Grabowski. 1993. \emph{Handbook of
Individual Differences, Learning, and Instruction}. Hillsdale, NJ:
Lawrence Erlbaum,.

\leavevmode\hypertarget{ref-kendal2018}{}%
Kendal, Rachel L., Neeltje J. Boogert, Luke Rendell, Kevin N. Laland,
Mike Webster, and Patricia L. Jones. 2018. {``Social Learning
Strategies: Bridge-Building Between Fields.''} \emph{Trends in Cognitive
Sciences} 22 (7): 651--65.
\url{https://doi.org/10.1016/j.tics.2018.04.003}.

\leavevmode\hypertarget{ref-key2019}{}%
Key, A. J. M., and S. J. Lycett. 2019. {``Biometric Variables Predict
Stone Tool Functional Performance More Effectively Than Tool-Form
Attributes: A Case Study in Handaxe Loading Capabilities.''}
\emph{Archaeometry} 61 (3): 539--55.
\url{https://doi.org/10.1111/arcm.12439}.

\leavevmode\hypertarget{ref-key2015}{}%
Key, Alastair J. M., and Christopher J. Dunmore. 2015. {``The Evolution
of the Hominin Thumb and the Influence Exerted by the Non-Dominant Hand
During Stone Tool Production.''} \emph{Journal of Human Evolution} 78
(January): 60--69. \url{https://doi.org/10.1016/j.jhevol.2014.08.006}.

\leavevmode\hypertarget{ref-key2018}{}%
---------. 2018. {``Manual Restrictions on Palaeolithic Technological
Behaviours.''} \emph{PeerJ} 6 (August): e5399.
\url{https://doi.org/10.7717/peerj.5399}.

\leavevmode\hypertarget{ref-key2014}{}%
Key, Alastair J. M., and Stephen J. Lycett. 2014. {``Are Bigger Flakes
Always Better? An Experimental Assessment of Flake Size Variation on
Cutting Efficiency and Loading.''} \emph{Journal of Archaeological
Science} 41 (January): 140--46.
\url{https://doi.org/10.1016/j.jas.2013.07.033}.

\leavevmode\hypertarget{ref-khreisheh2013}{}%
Khreisheh, Nada N., Danielle Davies, and Bruce A. Bradley. 2013.
{``Extending Experimental Control: The Use of Porcelain in Flaked Stone
Experimentation.''} \emph{Advances in Archaeological Practice} 1 (1):
38--46. \url{https://doi.org/10.7183/2326-3768.1.1.37}.

\leavevmode\hypertarget{ref-kline2015}{}%
Kline, Michelle Ann. 2015. {``How to learn about teaching: An
evolutionary framework for the study of teaching behavior in humans and
other animals.''} \emph{The Behavioral and Brain Sciences} 38: e31.
\url{https://doi.org/10.1017/S0140525X14000090}.

\leavevmode\hypertarget{ref-laland2017}{}%
Laland, Kevin N. 2017. {``The Origins of Language in Teaching.''}
\emph{Psychonomic Bulletin \& Review} 24 (1): 225--31.
\url{https://doi.org/10.3758/s13423-016-1077-7}.

\leavevmode\hypertarget{ref-levinson2014}{}%
Levinson, Stephen C., and Judith Holler. 2014. {``The Origin of Human
Multi-Modal Communication.''} \emph{Philosophical Transactions of the
Royal Society B: Biological Sciences} 369 (1651): 20130302.
\url{https://doi.org/10.1098/rstb.2013.0302}.

\leavevmode\hypertarget{ref-lombao2017}{}%
Lombao, D., M. Guardiola, and M. Mosquera. 2017. {``Teaching to Make
Stone Tools: New Experimental Evidence Supporting a Technological
Hypothesis for the Origins of Language.''} \emph{Scientific Reports} 7
(1): 1--14. \url{https://doi.org/10.1038/s41598-017-14322-y}.

\leavevmode\hypertarget{ref-marwick2017}{}%
Marwick, Ben. 2017. {``Computational Reproducibility in Archaeological
Research: Basic Principles and a Case Study of Their Implementation.''}
\emph{Journal of Archaeological Method and Theory} 24 (2): 424--50.
\url{https://doi.org/10.1007/s10816-015-9272-9}.

\leavevmode\hypertarget{ref-marzke1998}{}%
Marzke, Mary W., N. Toth, K. Schick, S. Reece, B. Steinberg, K. Hunt, R.
L. Linscheid, and K.-N. An. 1998. {``EMG Study of Hand Muscle
Recruitment During Hard Hammer Percussion Manufacture of Oldowan
Tools.''} \emph{American Journal of Physical Anthropology} 105 (3):
315--32.
\url{https://doi.org/10.1002/(SICI)1096-8644(199803)105:3\%3C315::AID-AJPA3\%3E3.0.CO;2-Q}.

\leavevmode\hypertarget{ref-miu2020}{}%
Miu, Elena, Ned Gulley, Kevin N. Laland, and Luke Rendell. 2020.
{``Flexible Learning, Rather Than Inveterate Innovation or Copying,
Drives Cumulative Knowledge Gain.''} \emph{Science Advances} 6 (23):
eaaz0286. \url{https://doi.org/10.1126/sciadv.aaz0286}.

\leavevmode\hypertarget{ref-molleman2019}{}%
Molleman, Lucas, Ralf H. J. M. Kurvers, and Wouter van den Bos. 2019.
{``Unleashing the BEAST: A Brief Measure of Human Social Information
Use.''} \emph{Evolution and Human Behavior} 40 (5): 492--99.
\url{https://doi.org/10.1016/j.evolhumbehav.2019.06.005}.

\leavevmode\hypertarget{ref-montagu1976}{}%
Montagu, Ashley. 1976. {``Toolmaking, Hunting, and the Origin of
Language.''} \emph{Annals of the New York Academy of Sciences} 280 (1):
266--74. \url{https://doi.org/10.1111/j.1749-6632.1976.tb25493.x}.

\leavevmode\hypertarget{ref-morgan2015}{}%
Morgan, T. J. H., N. T. Uomini, L. E. Rendell, L. Chouinard-Thuly, S. E.
Street, H. M. Lewis, C. P. Cross, et al. 2015. {``Experimental Evidence
for the Co-Evolution of Hominin Tool-Making Teaching and Language.''}
\emph{Nature Communications} 6 (1): 6029.
\url{https://doi.org/10.1038/ncomms7029}.

\leavevmode\hypertarget{ref-nonaka2010}{}%
Nonaka, Tetsushi, Blandine Bril, and Robert Rein. 2010. {``How Do Stone
Knappers Predict and Control the Outcome of Flaking? Implications for
Understanding Early Stone Tool Technology.''} \emph{Journal of Human
Evolution} 59 (2): 155--67.
\url{https://doi.org/10.1016/j.jhevol.2010.04.006}.

\leavevmode\hypertarget{ref-oakley1949}{}%
Oakley, Kenneth P. 1949. \emph{Man the Toolmaker}. London: Trustees of
the British Museum.

\leavevmode\hypertarget{ref-ohnuma1997}{}%
Ohnuma, Katsuhiko, Kenichi Aoki, and And Takeru Akazawa. 1997.
{``Transmission of Tool-Making Through Verbal and Non-Verbal
Commu-Nication: Preliminary Experiments in Levallois Flake
Production.''} \emph{Anthropological Science} 105 (3): 159--68.
\url{https://doi.org/10.1537/ase.105.159}.

\leavevmode\hypertarget{ref-pargeter2020}{}%
Pargeter, Justin, Nada Khreisheh, John J. Shea, and Dietrich Stout.
2020. {``Knowledge Vs. Know-How? Dissecting the Foundations of Stone
Knapping Skill.''} \emph{Journal of Human Evolution} 145 (August):
102807. \url{https://doi.org/10.1016/j.jhevol.2020.102807}.

\leavevmode\hypertarget{ref-pargeter2019}{}%
Pargeter, Justin, Nada Khreisheh, and Dietrich Stout. 2019.
{``Understanding Stone Tool-Making Skill Acquisition: {Experimental}
Methods and Evolutionary Implications.''} \emph{Journal of Human
Evolution} 133 (August): 146--66.
\url{https://doi.org/10.1016/j.jhevol.2019.05.010}.

\leavevmode\hypertarget{ref-pelegrin1990}{}%
Pelegrin, Jacques. 1990. {``Prehistoric Lithic Technology : Some Aspects
of Research.''} \emph{Archaeological Review from Cambridge} 9 (1):
116--25.
\href{https:///paper/Prehistoric-Lithic-Technology-}{/paper/Prehistoric-Lithic-Technology-}.

\leavevmode\hypertarget{ref-poldrack2011}{}%
Poldrack, Russell A. 2011. {``Inferring Mental States from Neuroimaging
Data: From Reverse Inference to Large-Scale Decoding.''} \emph{Neuron}
72 (5): 692--97. \url{https://doi.org/10.1016/j.neuron.2011.11.001}.

\leavevmode\hypertarget{ref-prasciunas2007}{}%
Prasciunas, Mary M. 2007. {``Bifacial Cores and Flake Production
Efficiency: An Experimental Test of Technological Assumptions.''}
\emph{American Antiquity} 72 (2): 334--48.
\url{https://doi.org/10.2307/40035817}.

\leavevmode\hypertarget{ref-putt2017}{}%
Putt, Shelby S., Sobanawartiny Wijeakumar, Robert G. Franciscus, and
John P. Spencer. 2017. {``The Functional Brain Networks That Underlie
Early Stone Age Tool Manufacture.''} \emph{Nature Human Behaviour} 1
(6): 1--8. \url{https://doi.org/10.1038/s41562-017-0102}.

\leavevmode\hypertarget{ref-putt2019}{}%
Putt, Shelby S., Sobanawartiny Wijeakumar, and John P. Spencer. 2019.
{``Prefrontal Cortex Activation Supports the Emergence of Early Stone
Age Toolmaking Skill.''} \emph{NeuroImage} 199 (October): 57--69.
\url{https://doi.org/10.1016/j.neuroimage.2019.05.056}.

\leavevmode\hypertarget{ref-putt2014}{}%
Putt, Shelby S., Alexander D. Woods, and Robert G. Franciscus. 2014.
{``The Role of Verbal Interaction During Experimental Bifacial Stone
Tool Manufacture.''} \emph{Lithic Technology} 39 (2): 96--112.
\url{https://doi.org/10.1179/0197726114Z.00000000036}.

\leavevmode\hypertarget{ref-rein2014}{}%
Rein, Robert, Tetsushi Nonaka, and Blandine Bril. 2014. {``Movement
Pattern Variability in Stone Knapping: Implications for the Development
of Percussive Traditions.''} \emph{PLOS ONE} 9 (11): e113567.
\url{https://doi.org/10.1371/journal.pone.0113567}.

\leavevmode\hypertarget{ref-reti2016}{}%
Reti, Jay S. 2016. {``Quantifying Oldowan Stone Tool Production at
Olduvai Gorge, Tanzania.''} \emph{PLOS ONE} 11 (1): e0147352.
\url{https://doi.org/10.1371/journal.pone.0147352}.

\leavevmode\hypertarget{ref-roux1995}{}%
Roux, Valentine, Blandine Bril, and Gilles Dietrich. 1995. {``Skills and
Learning Difficulties Involved in Stone Knapping: The Case of
Stone{-}Bead Knapping in Khambhat, India.''} \emph{World Archaeology} 27
(1): 63--87. \url{https://doi.org/10.1080/00438243.1995.9980293}.

\leavevmode\hypertarget{ref-rueden2017}{}%
Rueden, Curtis T., Johannes Schindelin, Mark C. Hiner, Barry E. DeZonia,
Alison E. Walter, Ellen T. Arena, and Kevin W. Eliceiri. 2017.
{``ImageJ2: ImageJ for the Next Generation of Scientific Image Data.''}
\emph{BMC Bioinformatics} 18 (1): 529.
\url{https://doi.org/10.1186/s12859-017-1934-z}.

\leavevmode\hypertarget{ref-sasaki2007}{}%
Sasaki, Hideo, Fumiyoshi Kasagi, Michiko Yamada, and Shoichiro Fujita.
2007. {``Grip Strength Predicts Cause-Specific Mortality in Middle-Aged
and Elderly Persons.''} \emph{The American Journal of Medicine} 120 (4):
337--42. \url{https://doi.org/10.1016/j.amjmed.2006.04.018}.

\leavevmode\hypertarget{ref-schillinger2014}{}%
Schillinger, Kerstin, Alex Mesoudi, and Stephen J. Lycett. 2014.
{``Copying Error and the Cultural Evolution of {``}Additive{''} Vs.
{``}Reductive{''} Material Traditions: An Experimental Assessment.''}
\emph{American Antiquity} 79 (1): 128--43.
\url{https://doi.org/10.7183/0002-7316.79.1.128}.

\leavevmode\hypertarget{ref-shallice1982}{}%
Shallice, Timothy, Donald Eric Broadbent, and Lawrence Weiskrantz. 1982.
{``Specific Impairments of Planning.''} \emph{Philosophical Transactions
of the Royal Society of London. B, Biological Sciences} 298 (1089):
199--209. \url{https://doi.org/10.1098/rstb.1982.0082}.

\leavevmode\hypertarget{ref-shea2015}{}%
Shea, John J. 2015. {``Making and Using Stone Tools: Advice for Learners
and Teachers and Insights for Archaeologists.''} \emph{Lithic
Technology} 40 (3): 231--48.
\url{https://doi.org/10.1179/2051618515Y.0000000011}.

\leavevmode\hypertarget{ref-shea2016}{}%
---------. 2016. \emph{Stone Tools in Human Evolution: Behavioral
Differences Among Technological Primates}. Cambridge: Cambridge
University Press. \url{https://doi.org/10.1017/9781316389355}.

\leavevmode\hypertarget{ref-sherwood2017}{}%
Sherwood, Chet C., and Aida Gómez-Robles. 2017. {``Brain Plasticity and
Human Evolution.''} \emph{Annual Review of Anthropology} 46 (1):
399--419. \url{https://doi.org/10.1146/annurev-anthro-102215-100009}.

\leavevmode\hypertarget{ref-stout2002}{}%
Stout, Dietrich. 2002. {``Skill and Cognition in Stone Tool Production:
An Ethnographic Case Study from Irian Jaya.''} \emph{Current
Anthropology} 43 (5): 693--722. \url{https://doi.org/10.1086/342638}.

\leavevmode\hypertarget{ref-stout2010}{}%
---------. 2010. {``Possible Relations Between Language and Technology
in Human Evolution.''} In, edited by April Nowell and Iain Davidson,
159184. Boulder, CO: University Press of Colorado.

\leavevmode\hypertarget{ref-stout2013}{}%
---------. 2013. {``Neuroscience of Technology.''} In, edited by Peter
J. Richerson and Morten H. Christiansen, 157--73. Cambridge, MA: The MIT
Press.

\leavevmode\hypertarget{ref-stout2014}{}%
Stout, Dietrich, Jan Apel, Julia Commander, and Mark Roberts. 2014.
{``Late Acheulean Technology and Cognition at Boxgrove, UK.''}
\emph{Journal of Archaeological Science} 41 (January): 576--90.
\url{https://doi.org/10.1016/j.jas.2013.10.001}.

\leavevmode\hypertarget{ref-stout2007}{}%
Stout, Dietrich, and Thierry Chaminade. 2007. {``The Evolutionary
Neuroscience of Tool Making.''} \emph{Neuropsychologia} 45 (5):
1091--1100.
\url{https://doi.org/10.1016/j.neuropsychologia.2006.09.014}.

\leavevmode\hypertarget{ref-stout2012}{}%
---------. 2012. {``Stone Tools, Language and the Brain in Human
Evolution.''} \emph{Philosophical Transactions of the Royal Society B:
Biological Sciences} 367 (1585): 75--87.
\url{https://doi.org/10.1098/rstb.2011.0099}.

\leavevmode\hypertarget{ref-stoutCognitiveDemandsLower2015}{}%
Stout, Dietrich, Erin Hecht, Nada Khreisheh, Bruce Bradley, and Thierry
Chaminade. 2015. {``Cognitive {Demands} of {Lower} {Paleolithic}
{Toolmaking}.''} \emph{PLOS ONE} 10 (4): e0121804.
\url{https://doi.org/10.1371/journal.pone.0121804}.

\leavevmode\hypertarget{ref-stout2015}{}%
Stout, Dietrich, and Nada Khreisheh. 2015. {``Skill Learning and Human
Brain Evolution: An Experimental Approach.''} \emph{Cambridge
Archaeological Journal} 25 (4): 867--75.
\url{https://doi.org/10.1017/S0959774315000359}.

\leavevmode\hypertarget{ref-stout2011}{}%
Stout, Dietrich, Richard Passingham, Christopher Frith, Jan Apel, and
Thierry Chaminade. 2011. {``Technology, expertise and social cognition
in human evolution.''} \emph{The European Journal of Neuroscience} 33
(7): 1328--38. \url{https://doi.org/10.1111/j.1460-9568.2011.07619.x}.

\leavevmode\hypertarget{ref-stout2005}{}%
Stout, Dietrich, Jay Quade, Sileshi Semaw, Michael J. Rogers, and Naomi
E. Levin. 2005. {``Raw Material Selectivity of the Earliest Stone
Toolmakers at Gona, Afar, Ethiopia.''} \emph{Journal of Human Evolution}
48 (4): 365--80. \url{https://doi.org/10.1016/j.jhevol.2004.10.006}.

\leavevmode\hypertarget{ref-stout2019}{}%
Stout, Dietrich, Michael J. Rogers, Adrian V. Jaeggi, and Sileshi Semaw.
2019. {``Archaeology and the Origins of Human Cumulative Culture: A Case
Study from the Earliest Oldowan at Gona, Ethiopia.''} \emph{Current
Anthropology} 60 (3): 309--40. \url{https://doi.org/10.1086/703173}.

\leavevmode\hypertarget{ref-stoutKnappingSkillEarliest2006}{}%
Stout, Dietrich, and Sileshi Semaw. 2006. {``Knapping Skill of the
Earliest Stone Toolmakers: Insights from the Study of Modern Human
Novices.''} In \emph{The {Oldowan}: {Case} Studies into the Earliest
{Stone} {Age}}, edited by Nicholas Toth and Kathy Schick, 307--20.
Gosport, IN: Stone Age Institute Press.

\leavevmode\hypertarget{ref-tehrani2008}{}%
Tehrani, Jamshid J., and Felix Riede. 2008. {``Towards an Archaeology of
Pedagogy: Learning, Teaching and the Generation of Material Culture
Traditions.''} \emph{World Archaeology} 40 (3): 316--31.
\url{https://doi.org/10.1080/00438240802261267}.

\leavevmode\hypertarget{ref-toth1993}{}%
Toth, Nicholas, and Kathy Schick. 1993. {``Early Stone Industries and
Inferences Regarding Language and Cognition.''} In, edited by Kathleen
R. Gibson and Tim Ingold, 346362. Cambridge: Cambridge University Press.

\leavevmode\hypertarget{ref-unsworth2005}{}%
Unsworth, Nash, and Randall W. Engle. 2005. {``Individual Differences in
Working Memory Capacity and Learning: Evidence from the Serial Reaction
Time Task.''} \emph{Memory \& Cognition} 33 (2): 213--20.
\url{https://doi.org/10.3758/BF03195310}.

\leavevmode\hypertarget{ref-washburn1960}{}%
Washburn, Sherwood L. 1960. {``Tools and Human Evolution.''}
\emph{Scientific American} 203 (3): 62--75.
\url{https://doi.org/10.1038/scientificamerican0960-62}.

\leavevmode\hypertarget{ref-whiten2015}{}%
Whiten, Andrew. 2015. {``Experimental Studies Illuminate the Cultural
Transmission of Percussive Technologies in Homo and Pan.''}
\emph{Philosophical Transactions of the Royal Society B: Biological
Sciences} 370 (1682): 20140359.
\url{https://doi.org/10.1098/rstb.2014.0359}.

\leavevmode\hypertarget{ref-whittaker1994}{}%
Whittaker, John C. 1994. \emph{Flintknapping Making and Understanding
Stone Tools By John C. Whittaker}. Austin, TX: University of Texas
Press.

\leavevmode\hypertarget{ref-wilkins2018}{}%
Wilkins, Jayne. 2018. {``The Point Is the Point: Emulative Social
Learning and Weapon Manufacture in the Middle Stone Age of South
Africa.''} In, edited by Michael J. O'Brien, Briggs Buchanan, and Metin
I. Eren, 153--74. Cambridge, MA: The MIT Press.

\leavevmode\hypertarget{ref-williams-hatala2018}{}%
Williams-Hatala, Erin Marie, Kevin G. Hatala, McKenzie Gordon, Alastair
Key, Margaret Kasper, and Tracy L. Kivell. 2018. {``The Manual Pressures
of Stone Tool Behaviors and Their Implications for the Evolution of the
Human Hand.''} \emph{Journal of Human Evolution} 119 (June): 14--26.
\url{https://doi.org/10.1016/j.jhevol.2018.02.008}.

\leavevmode\hypertarget{ref-wind2010}{}%
Wind, Anne E., Tim Takken, Paul J. M. Helders, and Raoul H. H.
Engelbert. 2010. {``Is Grip Strength a Predictor for Total Muscle
Strength in Healthy Children, Adolescents, and Young Adults?''}
\emph{European Journal of Pediatrics} 169 (3): 281--87.
\url{https://doi.org/10.1007/s00431-009-1010-4}.

\leavevmode\hypertarget{ref-wynn1979}{}%
Wynn, Thomas. 1979. {``The Intelligence of Later Acheulean Hominids.''}
\emph{Man} 14 (3): 371--91. \url{https://doi.org/10.2307/2801865}.

\leavevmode\hypertarget{ref-wynn2017}{}%
---------. 2017. {``Evolutionary Cognitive Archaeology.''} In, edited by
Thomas Wynn and Frederick Coolidge, 120. Oxford: Oxford University
Press.

\leavevmode\hypertarget{ref-wynn2004}{}%
Wynn, Thomas, and Frederick L. Coolidge. 2004. {``The expert Neandertal
mind.''} \emph{Journal of Human Evolution} 46 (4): 467--87.
\url{https://doi.org/10.1016/j.jhevol.2004.01.005}.

\leavevmode\hypertarget{ref-wynn2016}{}%
---------. 2016. {``Archeological Insights into Hominin Cognitive
Evolution.''} \emph{Evolutionary Anthropology: Issues, News, and
Reviews} 25 (4): 200--213. \url{https://doi.org/10.1002/evan.21496}.

\leavevmode\hypertarget{ref-yarkoni2020}{}%
Yarkoni, Tal. 2020. {``The Generalizability Crisis.''} \emph{Behavioral
and Brain Sciences}, 1--37.
\url{https://doi.org/10.1017/S0140525X20001685}.

\end{CSLReferences}

\bibliographystyle{spmpsci}
\bibliography{bibliography.bib}

\end{document}
