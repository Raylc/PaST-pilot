% !TeX program = pdfLaTeX
\documentclass[smallextended]{svjour3}       % onecolumn (second format)
%\documentclass[twocolumn]{svjour3}          % twocolumn
%
\smartqed  % flush right qed marks, e.g. at end of proof
%
\usepackage{amsmath}
\usepackage{graphicx}
\usepackage[utf8]{inputenc}

\usepackage[hyphens]{url} % not crucial - just used below for the URL
\usepackage{hyperref}
\providecommand{\tightlist}{%
  \setlength{\itemsep}{0pt}\setlength{\parskip}{0pt}}

%
% \usepackage{mathptmx}      % use Times fonts if available on your TeX system
%
% insert here the call for the packages your document requires
%\usepackage{latexsym}
% etc.
%
% please place your own definitions here and don't use \def but
% \newcommand{}{}
%
% Insert the name of "your journal" with
% \journalname{myjournal}
%

%% load any required packages here



% Pandoc citation processing


\begin{document}

\title{Testing the cognitive foundations of Paleolithic social
transmission \thanks{Grants or other notes about the article that should
go on the front page should be placed here. General acknowledgments
should be placed at the end of the article.} }



\author{  Justin Pargeter 1, 2 \and  Megan Beney Kilgore 3 \and  Cheng
Liu 3 \and  Dietrich Stout 3 \and  }


\institute{
        Justin Pargeter 1, 2 \at
     1. Department of Anthropology, New York University, New York, NY,
USA; 2. Palaeo-Research Institute, University of Johannesburg, Auckland
Park, South Africa \\
     \email{\href{mailto:justin.pargeter@nyu.edu}{\nolinkurl{justin.pargeter@nyu.edu}}}  %  \\
%             \emph{Present address:} of F. Author  %  if needed
    \and
        Megan Beney Kilgore 3 \at
     3. Department of Anthropology, Emory University, Atlanta, GA,
USA \\
     \email{\href{mailto:megan.elizabeth.beney@emory.edu}{\nolinkurl{megan.elizabeth.beney@emory.edu}}}  %  \\
%             \emph{Present address:} of F. Author  %  if needed
    \and
        Cheng Liu 3 \at
     3. Department of Anthropology, Emory University, Atlanta, GA,
USA \\
     \email{\href{mailto:raylc1996@outlook.com}{\nolinkurl{raylc1996@outlook.com}}}  %  \\
%             \emph{Present address:} of F. Author  %  if needed
    \and
        Dietrich Stout 3 \at
     3. Department of Anthropology, Emory University, Atlanta, GA,
USA \\
     \email{\href{mailto:dwstout@emory.edu}{\nolinkurl{dwstout@emory.edu}}}  %  \\
%             \emph{Present address:} of F. Author  %  if needed
    \and
    }

\date{Received: date / Accepted: date}
% The correct dates will be entered by the editor


\maketitle

\begin{abstract}
Stone tools provide key evidence of human cognitive evolution but remain
difficult to interpret. Toolmaking skill-learning, in particular, has
been understudied even though: 1) the most salient cognitive demands of
toolmaking should occur during learning, and 2) variation in learning
aptitude would have provided the raw material for any past selection
acting on tool making ability. Despite decades of research on stone
toolmaking we still know little about the cognitive prerequisites of
learning under different social transmission conditions that may have
prevailed during the Paleolithic. This paper presents results from a
pilot experimental study to trial new experimental methods for
investigating the effect of learning conditions and individual
differences on Oldowan flake-tool making skill acquisition. We trained
32 participants for 2 hours to make simple stone tools under two
different instructional conditions (observation-only vs.~direct-active
teaching) employing appropriate raw materials and in-person interaction.
Participant performance was evaluated through analysis of the stone
artifacts produced and was compared both across experimental groups and
with respect to individual participant differences in grip strength,
motor accuracy, and cognitive function measured for the study. Our
results show aptitude to be associated with fluid intelligence in a
verbally instructed group and with tendency to use social information in
an observation-only group. These results have implications for debates
surrounding the cumulative nature of human culture and the role of
evolved psychological mechanisms in ``high fidelity'' transmission of
information, particularly through imitation and teaching.
\\
\keywords{
        key \and
        dictionary \and
        word \and
    }


\end{abstract}


\def\spacingset#1{\renewcommand{\baselinestretch}%
{#1}\small\normalsize} \spacingset{1}


\hypertarget{intro}{%
\section{Introduction}\label{intro}}

Your text comes here. Separate text sections with \cite{Mislevy06Cog}.

\hypertarget{sec:1}{%
\section{Section title}\label{sec:1}}

Text with citations by \cite{Galyardt14mmm}.

\hypertarget{sec:2}{%
\subsection{Subsection title}\label{sec:2}}

as required. Don't forget to give each section and subsection a unique
label (see Sect. \ref{sec:1}).

\hypertarget{paragraph-headings}{%
\paragraph{Paragraph headings}\label{paragraph-headings}}

Use paragraph headings as needed.

\begin{align}
a^2+b^2=c^2
\end{align}

\bibliographystyle{spphys}
\bibliography{bibliography.bib}

\end{document}
