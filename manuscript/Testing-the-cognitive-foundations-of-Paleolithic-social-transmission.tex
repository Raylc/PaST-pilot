% !TeX program = pdfLaTeX
\documentclass[smallextended]{svjour3}       % onecolumn (second format)
%\documentclass[twocolumn]{svjour3}          % twocolumn
%
\smartqed  % flush right qed marks, e.g. at end of proof
%
\usepackage{amsmath}
\usepackage{graphicx}
\usepackage[utf8]{inputenc}

\usepackage[hyphens]{url} % not crucial - just used below for the URL
\usepackage{hyperref}
\providecommand{\tightlist}{%
  \setlength{\itemsep}{0pt}\setlength{\parskip}{0pt}}

%
% \usepackage{mathptmx}      % use Times fonts if available on your TeX system
%
% insert here the call for the packages your document requires
%\usepackage{latexsym}
% etc.
%
% please place your own definitions here and don't use \def but
% \newcommand{}{}
%
% Insert the name of "your journal" with
% \journalname{myjournal}
%

%% load any required packages here



% Pandoc citation processing
\newlength{\csllabelwidth}
\setlength{\csllabelwidth}{3em}
\newlength{\cslhangindent}
\setlength{\cslhangindent}{1.5em}
% for Pandoc 2.8 to 2.10.1
\newenvironment{cslreferences}%
  {}%
  {\par}
% For Pandoc 2.11+
\newenvironment{CSLReferences}[3] % #1 hanging-ident, #2 entry sp
 {% don't indent paragraphs
  \setlength{\parindent}{0pt}
  % turn on hanging indent if param 1 is 1
  \ifodd #1 \everypar{\setlength{\hangindent}{\cslhangindent}}\ignorespaces\fi
  % set line spacing
  % set entry spacing
  \ifnum #2 > 0
  \setlength{\parskip}{#3\baselineskip}
  \fi
 }%
 {}
\usepackage{calc} % for \widthof, \maxof
\newcommand{\CSLBlock}[1]{#1\hfill\break}
\newcommand{\CSLLeftMargin}[1]{\parbox[t]{\maxof{\widthof{#1}}{\csllabelwidth}}{#1}}
\newcommand{\CSLRightInline}[1]{\parbox[t]{\linewidth}{#1}}
\newcommand{\CSLIndent}[1]{\hspace{\cslhangindent}#1}


\begin{document}

\title{Testing the motor and cognitive foundations of Paleolithic social
transmission \thanks{Grants or other notes about the article that should
go on the front page should be placed here. General acknowledgments
should be placed at the end of the article.} }



\author{  Justin Pargeter 1, 2 \and  Megan Beney Kilgore 3 \and  Cheng
Liu 3 \and  Dietrich Stout 3 \and  }


\institute{
        Justin Pargeter 1, 2 \at
     1. Department of Anthropology, New York University, New York, NY,
USA; 2. Palaeo-Research Institute, University of Johannesburg, Auckland
Park, South Africa \\
     \email{\href{mailto:justin.pargeter@nyu.edu}{\nolinkurl{justin.pargeter@nyu.edu}}}  %  \\
%             \emph{Present address:} of F. Author  %  if needed
    \and
        Megan Beney Kilgore 3 \at
     3. Department of Anthropology, Emory University, Atlanta, GA,
USA \\
     \email{\href{mailto:megan.elizabeth.beney@emory.edu}{\nolinkurl{megan.elizabeth.beney@emory.edu}}}  %  \\
%             \emph{Present address:} of F. Author  %  if needed
    \and
        Cheng Liu 3 \at
     3. Department of Anthropology, Emory University, Atlanta, GA,
USA \\
     \email{\href{mailto:raylc1996@outlook.com}{\nolinkurl{raylc1996@outlook.com}}}  %  \\
%             \emph{Present address:} of F. Author  %  if needed
    \and
        Dietrich Stout 3 \at
     3. Department of Anthropology, Emory University, Atlanta, GA,
USA \\
     \email{\href{mailto:dwstout@emory.edu}{\nolinkurl{dwstout@emory.edu}}}  %  \\
%             \emph{Present address:} of F. Author  %  if needed
    \and
    }

\date{Received: date / Accepted: date}
% The correct dates will be entered by the editor


\maketitle

\begin{abstract}
Stone tools provide key evidence of human cognitive evolution but remain
difficult to interpret. Toolmaking skill-learning in particular has been
understudied even though: 1) the most salient cognitive demands of
toolmaking should occur during learning, and 2) variation in learning
aptitude would have provided the raw material for any past selection
acting on tool making ability. However, we actually know very little
about the cognitive prerequisites of learning under different
information transmission conditions that may have prevailed during the
Paleolithic. This paper presents results from a pilot experimental study
to trial new experimental methods for studying the effect of learning
conditions and individual differences on Oldowan flake-tool making skill
acquisition. We trained 23 participants for 2 hours to make stone flakes
under two different instructional conditions (observation only
vs.~direct active teaching) employing appropriate raw materials,
practice time, and real human interaction. Participant performance was
evaluated through analysis of the stone artifacts produced. Performance
was compared both across experimental groups and with respect to
individual participant differences in grip strength, motor accuracy, and
cognitive function measured for the study. Our results show aptitude to
be associated with fluid intelligence in a verbally instructed group and
with a tendency to use social information in an observation-only group.
These results have implications for debates surrounding the cumulative
nature of human culture, the relative contributions of knowledge and
know-how for stone tool making, and the role of evolved psychological
mechanisms in ``high fidelity'' transmission of information,
particularly through imitation and teaching.
\\
\keywords{
        Oldowan \and
        Stone toolmaking \and
        Social learning \and
        Individual variation \and
        Cognitive aptitudes \and
        Motor skills \and
    }


\end{abstract}


\def\spacingset#1{\renewcommand{\baselinestretch}%
{#1}\small\normalsize} \spacingset{1}


\hypertarget{intro}{%
\section{Introduction}\label{intro}}

Stone tools have long been seen as a key source of evidence for
understanding human behavioral and cognitive evolution (Darwin 1871;
Oakley 1949; Washburn 1960). Pathbreaking attempts to infer specific
cognitive capacities from this evidence largely focused on the basic
requirements of tool production (Isaac 1976; Wynn 1979; Gowlett 1984;
Wynn and Coolidge 2004). More recently, increasing attention has been
directed to the processes and demands of stone tool making skill
acquisition (Roux, Bril, and Dietrich 1995; Stout 2002; Stout et al.
2005; Geribàs, Mosquera, and Vergès 2010; Nonaka, Bril, and Rein 2010;
Stout et al. 2011; Putt, Woods, and Franciscus 2014; Hecht et al. 2015;
Duke and Pargeter 2015; Morgan et al. 2015; Stout and Khreisheh 2015;
Lombao, Guardiola, and Mosquera 2017; Putt et al. 2017; Cataldo,
Migliano, and Vinicius 2018; Putt, Wijeakumar, and Spencer 2019;
Pargeter and Shea 2019; Pargeter et al. 2020). This is motivated by the
expectation that the most salient cognitive demands of tool making
should occur during learning rather than routine expert performance
(Stout and Khreisheh 2015) and by interest in the relevance of different
social learning mechanisms such as imitation (Rein, Nonaka, and Bril
2014; Stout et al. 2019), emulation (Tehrani and Riede 2008; Wilkins
2018), and language (Ohnuma, Aoki, and Akazawa 1997; Putt, Woods, and
Franciscus 2014; Morgan et al. 2015; Lombao, Guardiola, and Mosquera
2017; Putt et al. 2017; Cataldo, Migliano, and Vinicius 2018) to the
reproduction of Paleolithic technologies.

\hypertarget{individual-differences}{%
\subsection{\texorpdfstring{\textbf{Individual
Differences}}{Individual Differences}}\label{individual-differences}}

\hypertarget{teaching-language-and-tool-making}{%
\subsection{\texorpdfstring{\textbf{Teaching, Language, and Tool
Making}}{Teaching, Language, and Tool Making}}\label{teaching-language-and-tool-making}}

\hypertarget{raw-materials-and-knapping-skill}{%
\subsection{\texorpdfstring{\textbf{Raw materials and knapping
skill}}{Raw materials and knapping skill}}\label{raw-materials-and-knapping-skill}}

\hypertarget{materials-and-methods}{%
\section{\texorpdfstring{\textbf{Materials and
Methods}}{Materials and Methods}}\label{materials-and-methods}}

\hypertarget{participants}{%
\subsection{\texorpdfstring{\textbf{Participants}}{Participants}}\label{participants}}

\hypertarget{study-visit}{%
\subsection{\texorpdfstring{\textbf{Study
Visit}}{Study Visit}}\label{study-visit}}

\hypertarget{individual-difference-measures}{%
\subsection{\texorpdfstring{\textbf{Individual Difference
Measures}}{Individual Difference Measures}}\label{individual-difference-measures}}

\hypertarget{stone-tool-making}{%
\subsection{\texorpdfstring{\textbf{Stone Tool
Making}}{Stone Tool Making}}\label{stone-tool-making}}

\hypertarget{raw-materials}{%
\subsubsection{\texorpdfstring{\textbf{Raw
Materials}}{Raw Materials}}\label{raw-materials}}

\hypertarget{experimental-conditions}{%
\subsubsection{\texorpdfstring{\textbf{Experimental
Conditions}}{Experimental Conditions}}\label{experimental-conditions}}

\hypertarget{lithic-analysis}{%
\subsection{\texorpdfstring{\textbf{Lithic
Analysis}}{Lithic Analysis}}\label{lithic-analysis}}

\hypertarget{statistical-analyses}{%
\subsection{\texorpdfstring{\textbf{Statistical
Analyses}}{Statistical Analyses}}\label{statistical-analyses}}

\hypertarget{results}{%
\section{\texorpdfstring{\textbf{Results}}{Results}}\label{results}}

\hypertarget{principal-component-analyses}{%
\subsection{\texorpdfstring{\textbf{Principal Component
analyses}}{Principal Component analyses}}\label{principal-component-analyses}}

\hypertarget{flake-size-and-shape}{%
\subsubsection{\texorpdfstring{\textbf{Flake size and
shape}}{Flake size and shape}}\label{flake-size-and-shape}}

\hypertarget{lithic-flaking-performance-measures}{%
\subsubsection{\texorpdfstring{\textbf{Lithic flaking performance
measures}}{Lithic flaking performance measures}}\label{lithic-flaking-performance-measures}}

\hypertarget{do-trained-untrained-and-expert-knappers-perform-differently}{%
\subsection{\texorpdfstring{\textbf{Do trained, untrained, and expert
knappers perform
differently?}}{Do trained, untrained, and expert knappers perform differently?}}\label{do-trained-untrained-and-expert-knappers-perform-differently}}

\hypertarget{does-trainingpractice-time-impact-flaking-performance}{%
\subsection{\texorpdfstring{\textbf{Does training/practice time impact
flaking
performance?}}{Does training/practice time impact flaking performance?}}\label{does-trainingpractice-time-impact-flaking-performance}}

\hypertarget{do-individual-differences-in-motor-skill-and-psychometric-measures-predict-flaking-performance}{%
\subsection{\texorpdfstring{\textbf{Do individual differences in motor
skill and psychometric measures predict flaking performance?
}}{Do individual differences in motor skill and psychometric measures predict flaking performance? }}\label{do-individual-differences-in-motor-skill-and-psychometric-measures-predict-flaking-performance}}

\hypertarget{model-1-individual-differences-and-quantity-flaking}{%
\subsubsection{\texorpdfstring{\textbf{Model 1: Individual differences
and quantity
flaking}}{Model 1: Individual differences and quantity flaking}}\label{model-1-individual-differences-and-quantity-flaking}}

\hypertarget{model-2-individual-differences-and-quality-flaking}{%
\subsubsection{\texorpdfstring{\textbf{Model 2: Individual differences
and quality
flaking}}{Model 2: Individual differences and quality flaking}}\label{model-2-individual-differences-and-quality-flaking}}

\hypertarget{discussion}{%
\section{\texorpdfstring{\textbf{Discussion}}{Discussion}}\label{discussion}}

\hypertarget{conclusions}{%
\section{\texorpdfstring{\textbf{Conclusions}}{Conclusions}}\label{conclusions}}

\hypertarget{acknowledgments}{%
\section{\texorpdfstring{\textbf{Acknowledgments}}{Acknowledgments}}\label{acknowledgments}}

\hypertarget{references}{%
\section*{\texorpdfstring{\textbf{References}}{References}}\label{references}}
\addcontentsline{toc}{section}{\textbf{References}}

\hypertarget{refs}{}
\begin{CSLReferences}{1}{0}
\leavevmode\hypertarget{ref-cataldo2018}{}%
Cataldo, Dana Michelle, Andrea Bamberg Migliano, and Lucio Vinicius.
2018. {``Speech, Stone Tool-Making and the Evolution of Language.''}
\emph{PLOS ONE} 13 (1): e0191071.
\url{https://doi.org/10.1371/journal.pone.0191071}.

\leavevmode\hypertarget{ref-darwin1871}{}%
Darwin, Charles. 1871. \emph{The Descent of Man, and Selection in
Relation to Sex}. 1st ed. London: John Murray.

\leavevmode\hypertarget{ref-duke2015}{}%
Duke, Hilary, and Justin Pargeter. 2015. {``Weaving Simple Solutions to
Complex Problems: An Experimental Study of Skill in Bipolar
Cobble-Splitting.''} \emph{Lithic Technology} 40 (4): 349--65.
\url{https://doi.org/10.1179/2051618515Y.0000000016}.

\leavevmode\hypertarget{ref-geribuxe0s2010}{}%
Geribàs, Núria, Marina Mosquera, and Josep Maria Vergès. 2010. {``What
Novice Knappers Have to Learn to Become Expert Stone Toolmakers.''}
\emph{Journal of Archaeological Science} 37 (11): 2857--70.
\url{https://doi.org/10.1016/j.jas.2010.06.026}.

\leavevmode\hypertarget{ref-gowlett1984}{}%
Gowlett, John A. J. 1984. {``Mental Abilities of Early Man: A Look at
Some Hard Evidence.''} \emph{Higher Education Quarterly} 38 (3):
199--220. \url{https://doi.org/10.1111/j.1468-2273.1984.tb01387.x}.

\leavevmode\hypertarget{ref-hecht2015}{}%
Hecht, E. E., D. A. Gutman, N. Khreisheh, S. V. Taylor, J. Kilner, A. A.
Faisal, B. A. Bradley, T. Chaminade, and D. Stout. 2015. {``Acquisition
of Paleolithic Toolmaking Abilities Involves Structural Remodeling to
Inferior Frontoparietal Regions.''} \emph{Brain Structure and Function}
220 (4): 2315--31. \url{https://doi.org/10.1007/s00429-014-0789-6}.

\leavevmode\hypertarget{ref-isaac1976}{}%
Isaac, Glynn L. 1976. {``Stages of Cultural Elaboration in the
Pleistocene: Possible Archaeological Indicators of the Development of
Language Capabilities.''} \emph{Annals of the New York Academy of
Sciences} 280 (1): 275--88.
\url{https://doi.org/10.1111/j.1749-6632.1976.tb25494.x}.

\leavevmode\hypertarget{ref-lombao2017}{}%
Lombao, D., M. Guardiola, and M. Mosquera. 2017. {``Teaching to Make
Stone Tools: New Experimental Evidence Supporting a Technological
Hypothesis for the Origins of Language.''} \emph{Scientific Reports} 7
(1): 1--14. \url{https://doi.org/10.1038/s41598-017-14322-y}.

\leavevmode\hypertarget{ref-morgan2015}{}%
Morgan, T. J. H., N. T. Uomini, L. E. Rendell, L. Chouinard-Thuly, S. E.
Street, H. M. Lewis, C. P. Cross, et al. 2015. {``Experimental Evidence
for the Co-Evolution of Hominin Tool-Making Teaching and Language.''}
\emph{Nature Communications} 6 (1): 6029.
\url{https://doi.org/10.1038/ncomms7029}.

\leavevmode\hypertarget{ref-nonaka2010}{}%
Nonaka, Tetsushi, Blandine Bril, and Robert Rein. 2010. {``How Do Stone
Knappers Predict and Control the Outcome of Flaking? Implications for
Understanding Early Stone Tool Technology.''} \emph{Journal of Human
Evolution} 59 (2): 155--67.
\url{https://doi.org/10.1016/j.jhevol.2010.04.006}.

\leavevmode\hypertarget{ref-oakley1949}{}%
Oakley, Kenneth P. 1949. \emph{Man the Toolmaker}. London: Trustees of
the British Museum.

\leavevmode\hypertarget{ref-ohnuma1997}{}%
Ohnuma, Katsuhiko, Kenichi Aoki, and And Takeru Akazawa. 1997.
{``Transmission of Tool-Making Through Verbal and Non-Verbal
Commu-Nication: Preliminary Experiments in Levallois Flake
Production.''} \emph{Anthropological Science} 105 (3): 159--68.
\url{https://doi.org/10.1537/ase.105.159}.

\leavevmode\hypertarget{ref-pargeter2020}{}%
Pargeter, Justin, Nada Khreisheh, John J. Shea, and Dietrich Stout.
2020. {``Knowledge Vs. Know-How? Dissecting the Foundations of Stone
Knapping Skill.''} \emph{Journal of Human Evolution} 145 (August):
102807. \url{https://doi.org/10.1016/j.jhevol.2020.102807}.

\leavevmode\hypertarget{ref-pargeter2019}{}%
Pargeter, Justin, and John J. Shea. 2019. {``Going Big Versus Going
Small: Lithic Miniaturization in Hominin Lithic Technology.''}
\emph{Evolutionary Anthropology: Issues, News, and Reviews} 28 (2):
72--85. \url{https://doi.org/10.1002/evan.21775}.

\leavevmode\hypertarget{ref-putt2017}{}%
Putt, Shelby S., Sobanawartiny Wijeakumar, Robert G. Franciscus, and
John P. Spencer. 2017. {``The Functional Brain Networks That Underlie
Early Stone Age Tool Manufacture.''} \emph{Nature Human Behaviour} 1
(6): 1--8. \url{https://doi.org/10.1038/s41562-017-0102}.

\leavevmode\hypertarget{ref-putt2019}{}%
Putt, Shelby S., Sobanawartiny Wijeakumar, and John P. Spencer. 2019.
{``Prefrontal Cortex Activation Supports the Emergence of Early Stone
Age Toolmaking Skill.''} \emph{NeuroImage} 199 (October): 57--69.
\url{https://doi.org/10.1016/j.neuroimage.2019.05.056}.

\leavevmode\hypertarget{ref-putt2014}{}%
Putt, Shelby S., Alexander D. Woods, and Robert G. Franciscus. 2014.
{``The Role of Verbal Interaction During Experimental Bifacial Stone
Tool Manufacture.''} \emph{Lithic Technology} 39 (2): 96--112.
\url{https://doi.org/10.1179/0197726114Z.00000000036}.

\leavevmode\hypertarget{ref-rein2014}{}%
Rein, Robert, Tetsushi Nonaka, and Blandine Bril. 2014. {``Movement
Pattern Variability in Stone Knapping: Implications for the Development
of Percussive Traditions.''} \emph{PLOS ONE} 9 (11): e113567.
\url{https://doi.org/10.1371/journal.pone.0113567}.

\leavevmode\hypertarget{ref-roux1995}{}%
Roux, Valentine, Blandine Bril, and Gilles Dietrich. 1995. {``Skills and
Learning Difficulties Involved in Stone Knapping: The Case of
Stone{-}Bead Knapping in Khambhat, India.''} \emph{World Archaeology} 27
(1): 63--87. \url{https://doi.org/10.1080/00438243.1995.9980293}.

\leavevmode\hypertarget{ref-stout2002}{}%
Stout, Dietrich. 2002. {``Skill and Cognition in Stone Tool Production:
An Ethnographic Case Study from Irian Jaya.''} \emph{Current
Anthropology} 43 (5): 693--722. \url{https://doi.org/10.1086/342638}.

\leavevmode\hypertarget{ref-stout2015}{}%
Stout, Dietrich, and Nada Khreisheh. 2015. {``Skill Learning and Human
Brain Evolution: An Experimental Approach.''} \emph{Cambridge
Archaeological Journal} 25 (4): 867--75.
\url{https://doi.org/10.1017/S0959774315000359}.

\leavevmode\hypertarget{ref-stout2011}{}%
Stout, Dietrich, Richard Passingham, Christopher Frith, Jan Apel, and
Thierry Chaminade. 2011. {``Technology, expertise and social cognition
in human evolution.''} \emph{The European Journal of Neuroscience} 33
(7): 1328--38. \url{https://doi.org/10.1111/j.1460-9568.2011.07619.x}.

\leavevmode\hypertarget{ref-stout2005}{}%
Stout, Dietrich, Jay Quade, Sileshi Semaw, Michael J. Rogers, and Naomi
E. Levin. 2005. {``Raw Material Selectivity of the Earliest Stone
Toolmakers at Gona, Afar, Ethiopia.''} \emph{Journal of Human Evolution}
48 (4): 365--80. \url{https://doi.org/10.1016/j.jhevol.2004.10.006}.

\leavevmode\hypertarget{ref-stout2019}{}%
Stout, Dietrich, Michael J. Rogers, Adrian V. Jaeggi, and Sileshi Semaw.
2019. {``Archaeology and the Origins of Human Cumulative Culture: A Case
Study from the Earliest Oldowan at Gona, Ethiopia.''} \emph{Current
Anthropology} 60 (3): 309--40. \url{https://doi.org/10.1086/703173}.

\leavevmode\hypertarget{ref-tehrani2008}{}%
Tehrani, Jamshid J., and Felix Riede. 2008. {``Towards an Archaeology of
Pedagogy: Learning, Teaching and the Generation of Material Culture
Traditions.''} \emph{World Archaeology} 40 (3): 316--31.
\url{https://doi.org/10.1080/00438240802261267}.

\leavevmode\hypertarget{ref-washburn1960}{}%
Washburn, Sherwood L. 1960. {``Tools and Human Evolution.''}
\emph{Scientific American} 203 (3): 62--75.
\url{https://doi.org/10.1038/scientificamerican0960-62}.

\leavevmode\hypertarget{ref-wilkins2018}{}%
Wilkins, Jayne. 2018. {``The Point Is the Point: Emulative Social
Learning and Weapon Manufacture in the Middle Stone Age of South
Africa.''} In, edited by Michael J. O'Brien, Briggs Buchanan, and Metin
I. Eren, 153--74. Cambridge, MA: The MIT Press.

\leavevmode\hypertarget{ref-wynn1979}{}%
Wynn, Thomas. 1979. {``The Intelligence of Later Acheulean Hominids.''}
\emph{Man} 14 (3): 371--91. \url{https://doi.org/10.2307/2801865}.

\leavevmode\hypertarget{ref-wynn2004}{}%
Wynn, Thomas, and Frederick L. Coolidge. 2004. {``The expert Neandertal
mind.''} \emph{Journal of Human Evolution} 46 (4): 467--87.
\url{https://doi.org/10.1016/j.jhevol.2004.01.005}.

\end{CSLReferences}

\bibliographystyle{spbasic}
\bibliography{bibliography.bib}

\end{document}
